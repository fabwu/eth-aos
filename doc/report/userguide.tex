\chapter{User guide}

\section{Shell}

To run the shell, the \verb|INIT_EXECUTE_SHELL| macro in \verb|usr/init/main.c| must be set to 1.

If an SD card is attached to the board, then the \verb|SDCARD_PRESENT| macro in \verb|usr/shell/main.c| can be set to 1.
If an SD card is not attached, then the macro must be set to 0.

\section{Networking}

To enable networking, the \verb|INIT_EXECUTE_ENET| macro in \verb|usr/init/main.c| must be set to 1.

\section{Filesystem}

Linux has the tendency to create vfat names if filename is not all uppercase.
This is especially important as the filesystem not properly recognize that it's
a vfat name and chockes on the zero byte terminator that the vfat name
necessarily contains.

To spawn a process off of the sdcard, one needs to first copy the binary onto
the sdcard, to make sure the binary is up to date, compatible with the current
code state.
The one needs to change \verb|INIT_EXECUTE_HELLO_SDCARD| macro in
\verb|usr/init/main.c| must be set to 1.
Also, the shell should be disabled for this (\verb|INIT_EXECUTE_SHELL|), as else
there might to initialisations of the filesystem.
We used the hello binary, if a different binary is to be spawned, the path
under the should be changed appropriatly.
