\documentclass[11pt,a4paper]{report}

\usepackage[english]{babel}

\usepackage[hidelinks,unicode]{hyperref}
\hypersetup{
    pdftitle={Advanced Operating Systems - Report},
    pdfauthor={
        Fabian Wüthrich,
        Janis Peyer,
        Kristina Martšenko,
        Philippe Mazenauer
    }
}

\usepackage{csquotes}
\usepackage{biblatex}
\addbibresource{main.bib}

\usepackage{listings}
\usepackage{bytefield}
\usepackage{xcolor}

\definecolor{codegreen}{rgb}{0,0.6,0}
\definecolor{codegray}{rgb}{0.5,0.5,0.5}
\definecolor{codepurple}{rgb}{0.58,0,0.82}
\definecolor{backcolour}{rgb}{0.95,0.95,0.92}

\lstdefinestyle{mystyle}{
    backgroundcolor=\color{backcolour},   
    commentstyle=\color{codegreen},
    keywordstyle=\color{magenta},
    numberstyle=\tiny\color{codegray},
    stringstyle=\color{codepurple},
    basicstyle=\ttfamily\footnotesize,
    breakatwhitespace=false,         
    breaklines=true,                 
    captionpos=b,                    
    keepspaces=true,                 
    numbers=none,                    
    numbersep=5pt,                  
    showspaces=false,                
    showstringspaces=false,
    showtabs=false,                  
    tabsize=2
}
\lstset{style=mystyle}

% For sequence diagram
\usepackage{tikz}
\usepackage{pgf-umlsd}
\usepgflibrary{arrows}

\begin{document}

\title{Advanced Operating Systems - Report}
\author{
Fabian W{\" u}thrich
\and
Janis Peyer 
\and
Kristina Mart{\v s}enko
\and
Philippe Mazenauer
}
\date{Spring 2020}

\maketitle

\tableofcontents

\listoffigures

\listoftables

\clearpage

% One chapter, but easier to handle this way
\chapter{Memory management and capabilities}

%123 \cite{aos-book}

The memory manager manages ranges of physical address space, we started to call
them memory ranges.
If one compares with paging, there there is a manager of ranges of virtual
address space, called addr\_mgr.
What both have in common is the invariant that ranges are to be disjunct, and
that a certain range should only be given out once, from the managers view there
exists only one owner of a certain range.
That's why both have quite similiar code, and one might have found a common
abstraction to refactor out, get the memory manager to use the addr\_mgr.
But the memory manager must also cope with the fact that there exists a physical
representation of it's ranges, the ram capabilites. And it needs to split them
according to a given request.

Ram capabilities are the foundation on which everything else (except device
frame), is built upon in the barrelfish world.
One needs a ram cap to create cnodes to store capabilites.
One needs ram caps to store the shadow paging tables.
One needs ram caps to for allocators.

\section{Access Control}

From an access control point of view, the memory manager does the following thins:
\begin{itemize}
	\item Propagation, in that it copies capabilities to send over lmp
	\item Restriciton, it splits the capabilities received from the boot info
	\item Amplification, ordinary processes don't have the boot info capabilities
		 		use RPC to get ram caps
\end{itemize}

\section{How}

The init process receives via the second argument (\verb|argv[1]|) the bootinfo
structure.
This structure contains information about the memory regions at \verb|regions|.

We iterate through this array of memory regions, and search for a memory region
of type \verb|RegionType_Empty| (which signifies its a ram cap).
We pass this to \verb|usr/init/mem_alloc.c:initialize_ram_alloc|.

This initialises the memory manager (\verb|lib/mm/mm.c:mm_init|).
An overview over the datastructures initialised is given in \ref{mem-data}.
We need to initialiase the three slab allocator used in the memory manager.
Also it zero initalises the member of \verb|struct mm|, so that we would catch
errors early.
Additionally, we initialise a mutex (\verb|alloc_mutex|), which is needed later
on when guarding against concurrency bugs.

After that, we grow each slab with statically allocated arrays (with space for
32 \verb|bi_node|'s, 64 \verb|mm_node|'s, and 64 \verb|aos_avl_node|'s).
This should be enough to satisfy to potentially add upto 32 memory regions,
and then also allow for refilling when serving alloc requests ($64-32=32$).
Why a reserve of 32 is needed is explained later in \ref{mem-con}.

The memory manager is now initialised to a point where we can pass it memory
regions via \verb|lib/mm/mm.c:mm_add|.
Compared to free, there two additionally thing to be considered when adding the
memory regions to the memory manager compared to a normal free of a memory
range.
They are not yet part of the \verb|all| double linked list.
This list contains all memory range in ascending order of their base address.
When adding memory regions them to it, we need to consider that the memory
ranges are not necessarily added in ascending order of their base address.
Additionally, from memory regions are the origin nodes (\verb|bi_node|) created
and added it to the \verb|bi| double linked list.
After having done that, the ram cap from the boot info has been transformed into
an ordinary memory range, which references the origin node, and can be added to
the free memory ranges using \verb|lib/mm/mm.c:mm_add_to_free|.

With all this done, the memory manager can now serve alloc and free requests.

\section{Datastructures} \label{mem-data}

To understand how alloc and free requests are served we introduce the
datastructures used in the memory manager.

\subsection{History}

Originally, everything has been implemented using two double linked lists (can
be seen on \verb|milestone2|).
One for \verb|struct capnode|, which stored the origin ram capabilities.
And one for \verb|struct mmnode|, which stored the memory ranges.
The memory ranges are stored in ascending order, and all of them, irrespective
if free or allocated.
Every memory range has a reference to a capnode.

When now \verb|lib/mm/mm.c:mm_alloc_aligned| is called, we would search the
mmnode linked list from the beginning, until we found a free node which is
greater than the size requested.
We split the mmnode from the left.
As free mmnode don't have a capability yet created for them, we don't need to
call \verb|cap_destroy|, but can go right for \verb|cap_retype|.
For this we use the reference to the capnode, because the capnode has the
original base, and so the \verb|cap_retype| is quite easy to do, as there are
no indirections, all ram capabilities handed out are direct descendants.

For \verb|lib/mm/mm.c:mm_free|, we again search the mmnode linked list, and we
expect to find an allocated mmnode with the base address given to free.
We mark this node simply as free.

Both of these operations have a cost of $\mathcal{O}(n)$, so this way of
implementing it is quite expensive.
Additionally, any external fragmentation isn't undone, even though the
datastructures have been setup to make this easy, by having memory ranges in
ascending order and using doubly linked list.

\subsection{Current State}

The current state still hase a double linked list of all memory ranges in
ascending order. This to allow for easy splitting (insertion) and fusing of
memory ranges (deletion), which are both $\mathcal{O}(1)$ operations on a doubly
linked list.

For allocation, we change from using the first memory range that's larger than
the requested size, to using the memory range with least size, which is greater
than the requested size.

As in \verb|lib/collections| the is no datastructure with allows for this
request to be fulfilled (hash table only allows for lookup of known keys), we
implemented an avl tree.
We chose the avl tree because we know it from the lectures, and didn't find
convincing literature that for example a red black tree is significantly better.
% TODO: Potentially add Performance Analysis of BSTs in System Software as
% source
The implementation of the avl tree is done so it doesn't allocate memory itself
(important later on, \ref{mem-con}).
If using slab allocator per avl tree, the avl nodes shouldn't be too dispersed
in memory, in theory making the pointer chasing somewhat cache friendly.
Also, it's versatile, in that it just stores opaque pointers (\verb|void *|),
allowing for reuse later on.
Additionally, when we have a reference to the specific avl node, we save one
lookup, making things a bit faster but the implementation also more complicated,
because we need to swap nodes under the hood (easy to get wrong).

So for allocation we have an avl tree that is indexed by the size of a memory
range, and as nodes double linked lists of memory ranges of that size (as it
can be that multiple free memory ranges have the same size).
Potentially the memory range is still to large, so we split it and add back to
the free avl tree ($\mathcal{O}(log(n))$), and insert into the all doubly linked
list ($\mathcal{O}(1)$).
As any of these memory range is ok, removing from this list is $\mathcal{0}(1)$.
With this, allocation is now an $\mathcal{O}(log(n))$ operation, which for worst
case performance is the theoretical optimum (comparison based search).

For freeing we need another avl tree indexed by the base address, the allocated
avl tree.
There is only ever one memory range with a certain base address, so here we
don't need a list as node.
We remove the memory range from the allocated tree.
Having a doubly linked list of memory ranges in ascending orders, and references
to the capnodes comes in handy now.
We can just check if the left neighbour is also free, and has same origin, and
fuse them together in $\mathcal{O}(log(n))$ (we need to reinsert into avl free
tree).
Vice versa for the right neighbour.

To note in all of this is that we do only one
\verb|cap_retype|/\verb|cap_destroy| per alloc/free, minimizing context
switches.
E.g. free memory ranges don't have their ram capabilites eagerly, because they
might be fused later on.
And additionally there is no hierarchy of ram capabilities to keep track of,
making \verb|cap_retype| simple to implement, just subtract the base of capnode
and one is set.

% TODO: Potentially more material
% 
% Datastructures of memory manager:
% \begin{itemize}
% 	\item Linked list of boot info ram chunks, provides the capabilities off which
% 				everything else is split, not intermediaries
% 	\item Linked list of all memory ranges, in ascending order of adderss
% 	\item Avl tree of free memory ranges, indexed by their size
% 	\item Avl tree of allocated memory ranges, indexed by their address
% \end{itemize}
% 
% The linked list of all nodes in ascending order of address can be managed with
% O(1) operations, and allows for easy fusing of memory ranges, if for example an
% allocated memory range gets freed, and there is an adjacent free on.
% 
% The avl tree of allocatee memory ranges, indexed by their address, is need
% because we only get the adress to free, but not the size, so thats for
% remembering the size of a certain memory range.
% 
% The avl tree of free memory ranges, indexed by ther size, allows for fast
% lookup when allocating memory, one just for the least element that's greater or
% euqal to the size requested.
% 
% Steps taken in implementation:
% \begin{itemize}
% 	\item Used multiple linked list initially to manage memory.
% 				Which was actually not too bad, because if memory is not freed, we can
% 				just chop off of the large object, having allocation in O(1), better than
% 				with trees (But only if the threes are badly done, as in this case the
% 				free avl tree also only contains one node, is basically also O(1)).
% 				Freeing is whole other story, as searching for the address to free is O(n).
% 	\item Changed later on to avl tree, when paging became too slow, memory was first to switch
% 	\item Fuses memory with adjacent free memory upon return
% \end{itemize}

\section{Fragmentation}

Every allocation request is rounded up to BASE\_PAGE\_SIZE.
This has multiple reasons.
Most ram will get mapped, and for it be able to be mapped, it must be multiple
of BASE\_PAGE\_SIZE.
So that's why we don't expect many requests that are note a multiple of
BASE\_PAGE\_SIZE, and therefore not having too much internal fragmentation.
Also, we observed that we only get \verb|lib/mm/mm.c:alloc_aligned| requests for
aligments smaller or equal to \verb|BASE_PAGE_SIZE|.
By having allocations being a mulitple of \verb|BASSE_PAGE_SIZE|, we don't
need to concern us further with alignment restrictions, as these are
automatically fulfilled.

For external fragmentation, we try to undo it by fusing adjacent free memory
ranges.
We try to use the smallest memory range that is large enough to fulfill the
memory request, and splitting off from the left, leaving as much of the free
memory rang intact.
This makes answering if there is a large enough memory range simple, as what
can be fused is fused.
We did not further investigate if this is detrimental to external fragmentation.

% TODO: Potentially more material
% 
% Fragmentation:
% \begin{itemize}
% 	\item Allocations always multiples of BASE\_PAGE\_SIZE
% 	\begin{itemize}
% 		\item Easy to fulfill requests with alignment <= BASE\_PAGE\_SIZE
% 		\item Reduces number of differen size held in avl tree of free memory ranges
% 		\item Should also reduce the amount of external fragmentation into the system
% 	\end{itemize}
% \end{itemize}
% 
% % We painted us a bit in a corner by not requiring to have capability as an
% % argument to mm_free
% Problems with lib/aos/capabilites.c:frame\_create, because it destroys the ram\_cap
% immediately. Not really, we could just as well have passed through frame cap.
% 
% Memory server, deletes cap as soon as sent off, not requiring cap makes for lighter
% free.

\section{Reentrancy and Concurrency}\label{mem-con}

The memory manager needs to be able to pull himself out of a ditch by it's own
hair (see Münchhausen).
To do that, it needs to always have enough ressources in reserve to build out
its datastructures without needing to build out its datastructures.

That's we use separate \verb|slab_allocator|'s, which we refill when we reach a
certain watermark.
A watermark of 32 nodes each for the tree and all list has so far proven to be
enough.
We werent able to verify this reserve staticly because of the complexity of the
\verb|lib/aos/slot_alloc| construct, especially with it's
\verb|twolevel_slot_alloc.c|.
That's also the reason why we only use the global slot\_allocator, not like with
the slabs, where we have separate slabs for the memory manager.
via \verb|lib/mm/mm.c:mm_add|.
Also that watermark refilling needs to be locked, and needs to be less or equal,
or else the refilling might not trigger early enough, or multiple times.
For the \verb|slot_allocator|, we assume it itself keeps enough reserves to build
itself, as it doesn't provide any method to check it's free slots, allows for
explicit refill.

Also, to build out it's datastructures, it needs to call \verb|ram_alloc| while
it's still in a ram alloc.
E.g. the code needs to handle reentrancy, a form of concurrency, correctly.

For that we need to identify sections of code, so called critical sections, that
can't handle being preemted, another request running through concurrently, then
resuming.
This is for example the case when we for alloc have a suiting avl node, then
would get preemted, get that suiting avl node snatched away, but we being
oblivious to that, stil continue with that avl node, corrupting the
datastructures.

Having identified the critical sections, we move out all calls that might to
reentrancy, for an isr that would be equivalent to disable interrupts.
That's why having avl trees not manage their own memory is so critical,
otherwise we wouldn't be able to write correct code.
To make sure we didn't miss a call, and to explicitly state this invariant, the
critical sections are guarded by the \verb|alloc_mutex|.
We call this type of not explicitly needed mutexes a canary mutex, and like a
canary in a mine, it would warn us if something changed in the system that we
did not anticipate, by leading to a deadlock, locking up the system.
It would also have been nice if lib/aos/thread\_sync.c:thread\_mutex\_lock would
have an assert checking if the same thread id retakes the lock, which would
always be a deadlock condition.

% --- End ---
% 
% A reserve of 32 has so far deemed to be large enough to serve every
% slab/slot\_allocator refill recursion loop we have encountered
% 
% Slab\_alloc can't trigger reentrancy, because it is explicitly filled. Slot\_alloc
% in contrast can trigger unwanted reentrancy.
% 
% Have lock which guarantees invariant that critical sections are reentrancy free.
% If not, we get a deadlock, and have an early warning that our critical section
% is not reentrancy free anymore.

\section{Limitations}

We only support for at most one memory region coming from the boot info.

In case of an error, we might not fully clean up after us.
This shouldn't invalidate our invariants, but might reduce the amount of memory
effectively usable.

We do not get a ram cap when freeing, which was already so in the original hand
out.
So we are not sure when freeing ram if the one freeing is really owner of the
freed memory range.

The implementation should work correctly until \verb|slab_default_refill| errors,
because the code assumes \verb|slab_alloc| always works, this would trigger an
assert on a later alloc.


\section{Address manager}

\subsection{How/Datastructures}

The addr manager does what the memory manager does for physical addresses for
virtual addresses.

It is initialised by giving it a minimal addr, and a max addr.
This means that it doens't need to handle physical representation (ram
capabilities), so the bi linked list and origin references are not needed anymore.

But it needs to additionally allow for requesting specific virtual addresses
(\verb|addr_mgr_alloc_fixed|).
To allow for this, we not only need have an avl tree for free ranges indexed by
size, but additionally another avl tree with the same free ranges indexed by
their base address.
For that wee need to find the range with largest base address, which is smaller
or equal to the requested range base address.
We then need to check if the range from the requested base address onwards is
large enough, and carve out the desired range by splitting it to two times into
smaller ranges.

To allow for checking if a virtual address is really in use, or this is just
dereferencing a stray pointer, we also allow to query if there exists an
allocated range for an address (\verb|addr_mgr_is_addr_allocated|).

With that, we have all the functionalities needed by paging.

\subsection{Limitations}

As should be obvious by now, the code of the address manager and memory manager
is mostly duplicate, it should be possible to roll it into some general
abstraction used by both paging and memory manager.

\section{Paging}

Paging manages the page table, and in doing that interacts with the address
manager to check if what it is going to page is valid.
Only after paging is memory given out by the memory manager really usable to
applications, as they are normally not allowed to access physical memory
directly.
We already see that paging and memory manager are interdependent, as
paging can't page without getting memory from memory manager, and the memory
manger can't give out memory without having its buffer paged in.

\subsection{How}

Paging is intialised by calling \verb|paging_init|.
As we don't get passed through any of the page tables capabilities for the
addresses already paged, we need to set the start address of the address manager
to the second slot of the l0 page table.
This ensures we don't overwrite the first slot of the l0 page table, wiping out
our existing virtual adress space.
We hope that $2^39$ bytes should be enough for the existing address space, it
doesn't use any of the other l0 slots except for the first.

We then call \verb|paging_init_state|, which stores the l0 page table capility,
initialises the mutexes, and slabs, and also initalises the address manager.
We then grow the slabs with slabs with static buffers, as for the first few
pagings we can't depend upon memory manager, we would get into a circular
dependency.

\subsection{Datastructures}

\begin{displayquote}
Turtles all the way down.
\end{displayquote}

We describe here how we implemented the shadow page table.

To note here is that we chose not to go for a one to one shadow page table.
This because we not only need to store the address of the next lower level
shadow page table, but also the mapping capability, the table capablity. So we
wouldn't fit into on page, which would have made things quite complicated,
inefficient.

\subsubsection{History}

Originally, every thing was doubly linked list (a bit of recurring theme). 
\begin{itemize}
	\item l0 would be a paging node without a parent, its page table capability,
				and a doubly linked list of its children.
	\item l1 would be paging node with l0 as a parent, its page table and mapping
				capability, and a doubly linked list of its children.
	\item l2 would be paging node with l1 as a parent, its page table and mapping
				capability, and a doubly linked list of its children.
	\item l3 would be paging node with l2 as a parent, its page table and mapping
				capability, and a doubly linked list of its children.
	\item l4 would be paging node with l3 as a parent, without page table and mapping
				capability, but instead it's page table capability would be set to the
				frame capability. Also, it's children would also be \verb|NULL|.
\end{itemize}

Every level except for l0 has a slot assigned to it. Also we store the level at
which the paging node is, making it easier to deduplicate code, handle all the
different levels mostly the same.

The address manager didn't exist yet. Therefore in \verb|paging_map_frame_attr|
there would just be a static counter initialised to zero, incremented according
to how many pages need to be mapped to fulfill mapping request.

\subsubsection{Current State}

In the current state we replaced the doubly linked list with avl trees, but
keeping the same overall structure, code.
This gives quite a speed up, as instead of invoking four times an $\mathcal{O}(n)$
operation (keeping in mind that n is bounded by $2^9$), we have four times
$\mathcal{O}(log(n))$. For the worst case ($n=2^9$), this gives a speed of up
two magnitudes for mapping one pag (it would be obvious for the code to not
retraverse the wole page table when mapping multiple pages at once, but as we
will see in \ref{pag-con}, this is not feasible, so we hit this speed up for
ever \verb|BASE_PAGE_SIZE| of bytes mapped).

And we switched out the static counter for the address manager, as one simple
counter wouldn't fulfill the requirements anymore (see later chapters).

\subsection{Reentrancy and Concurrency} \label{pag-con}

First to note again that we need to break up the interdependency of paging and
the memory manager, e.g. Münchhausen again.
This means paging needs to be able to page without the need to page in internal
datastructures.
This means memory allocator needs to be able to serve memory without the need to
use memory for internal datastructures.
That's why both use slab allocators, which we refill when we are below a
watermark, and have bit that tells us if we are already refilling up again, to
not double refill, get into an infinite recurse.

And we might get quite deep dependency chains, with for example paging wanting
to refill its slabs, because of that memory manager needing to allocate a slot,
the slot allocator needing memory to expand the cspace, and finally the memory
manager needing to page because of of it refilling its slabs, needing to page in
memory for that (probably not even the deepest it can go).

As in memory manager, we also have just emperically decided for a
reserve/watermark of 32 for the slabs.

But importantly this is bounded number, we can't have unlimited reserve
ressources.

Also to note is we allow for unmapping, so page tables might also get delete.

Why the unmap is decisive for the critical region, is that we might
(actually not sure about this, we did not investigate further into this because allow for
unmapping)
get away with fairly simple locking when adding child page table and the
critical region is only spanning one page table level
(E.g. we want add to a page table into certain slot, but someone else already
did that.
Especially for higher page table levels there is quite some contention.)

When we now introduce that page tables also might get deleted, the page table we
are about to add to might just have been pulled from under our feet.
So we would need to have full blown concurrent data structures (again, we think
this is possible, but we did not investigate further), which would have
been "fun" to do, but we chose not to go down that path.

So that forces us to have a critical regions that spans the full page table
walk.

But we also have upper bound for the critical region which are sure about.
We might want to cache the results of the page table walk to speed up paging.
But as we just argued before, this would increase our critical region.
Therefore, caching of the tree traversal results to speed up further mapping
requests (potentially unbounded many) would lead to an unbounded size of the
critical region, the need for unbounded amounts of reserve ressources.
That's why we chose not to cache results of the page walk.

\subsubsection{History}

Initially, the counter was used as is, and increment after paging.
As one can imagine, if we trigger reentrancy by refilling the slabs, we did
overwrite what we just did write, leading to interesting bugs.
This was one of the first encounters with reentrancy, as it is always triggerd
on the first paging, and was hard to reason as one didn't know yet what was one
dealing with.

That's why the counter is read into a temporary variable, then incremented, only
then we call page in.
This guarantees that the counter values for the different paging calls are
disjunct.

\subsection{Limitations}

We could have cached the page table walk results for fixed number of lookups,
and increase the ressource reserves accordingly.

We don't support mapping of huge pages, which could give quite some speedup, as
significantly less context switches.


\chapter{Processes, threads, and dispatch}

Before the Barrelfish kernel will spawn a process for us, a lot of preparation in userspace
has to be done. We have to create the CSpace and virtual address space for the new domain, load
the binary and setup the dispatcher. Whereas, all tools needed for creating the CSpace where provided
by the handout, we had to extend the paging system to create the VSpace for the child. 

At the beginning of this milestone the paging was able to map a frame at a given virtual address. 
Our first task was to extend the paging code such that we can map a frame at any free virtual address.
All important changes and design decision about paging are documented in section \ref{sec:paging} and
are not discussed here. The only relevant design decision for this chapter is, that the paging library
doesn't handle unaligned addresses. This was a deliberate choice to make the client actively think about
what they actually map. One of these clients (a certain group member) wasn't thinking hard enough about
his mappings, which caused a longer debugging session, but more on that later.

Although, the workload of this milestone was quite heavy, there weren't much design decisions involved.
All we had to do was following a receipt and spawn a process in the end. For that reason is this chapter
relatively short.

After we extended the paging code we had to locate the ELF binary. This was relatively easy as most code
was given by the handout. Now that we have a frame capability to the binary, it is time to map it into
\verb|init|'s address space using the recently implemented \verb|paging_map_frame_attr()|.

Next we had to setup the CSpace of the new domain. This was pretty straight forward, although the whole
capability system was a bit intimidating at first sight.

In order to setup the virtual address space for the child, we create a new paging state with 
\verb|paging_init_state_foreign|. We extended this function with the argument \verb|max_addr| to specify
an upper limit on the virtual address space. The child doesn't know anything about the VSpace, which
\verb|init| had created before (passing the paging state is marked as extra challenge). Thus, we use
only addresses from from slot 0 of the level 0 page table to setup the VSpace in \verb|init|. We can set
this limit with the \verb|max_addr| argument. When the child starts running, the paging is initialized to
hand out addresses from slot 1 of the level 0 page table onward. This prevents any conflicts, with the 
mappings setup by \verb|init|.

Now that we have paging state for the child, we can start mapping frames into the child's address space.
Most mappings were straightforward, only the mapping of the ELF segments caused some troubles. The 
paging library expects correctly aligned addresses from the client and the addresses for the ELF sections
where obviously not correctly aligned. This was one reason why we initially failed to spawn a process, but
after clearly communicating this requirement, we mapped the ELF segments at the correct location.

The next bit was parsing the commandline arguments. We did just the standard parsing without any sophisticated
features like escaping. Then we initialized the dispatcher structure, set the correct entry point and finally
called \verb|armv8_set_registers()| to spawn the process and, of course, nothing happened. First we fixed the
issue with the ELF segments and then debugged the kernel to find out why the new process wasn't coming up. It
turned out that we had to set register \verb|x0| to the address of the argument frame and after we did that
the process finally printed a \verb|Hello, world! from userspace|.

It is not mentioned anywhere in the book, that we have to set register \verb|x0| correctly, so we are not
sure if this was omitted deliberately or by mistake. Anyway, it was a valuable lesson to debug in a very
limited environment like kernel.



\chapter{(Lightweight) Message Passing}

Documented here is how we designed and implemented the LMP (Lightweight Message Passing) during milestone 3 and some of the improvements we made later on. The changes made in the individual project nameserver are documented in their own chapter. The nameserver project made an overhaul on the LMP and the final code base might look substentially different than what is documented here.

\section{Architecture / Design}

\subsection{Init Monolith vs. Many Services}

One of the first big design decisions in this milestone was, where to run the different services. There were the terminal service and memory service that had to be implemented during this milestone and we knew that later on we had to have a process management service. We considered two options: Running all of those services in the init domain or separating them and starting an own domain for each service.

Disadvantages for a monolithic init:
\begin{itemize}
    \item Complexity: Having all the services in init instead of separating them means more complexity. The interfaces between different services are not that well defined and it is harder to maintain and test a monolith.
\end{itemize}

Advanteges for a monolithic init:
\begin{itemize}
    \item Performance: Having all the services run in the same domain and on the same thread means less context switches and when staying in init no context switches at all.
    \item Easier to develop: Adding the services to init meant we did not have to rewrite any of the working functionality inside of init but could just provide an interface for other applications.
\end{itemize}

We decided to implement the monolithic init. We did so mainly because we considered it way easier and faster to implement. Moving for example the memory service out of init meant that we would request memory from that service in init and we were at that point not confident that we could handle all the entailing problems in time for the milestone submission.

\section{Channel Setup - Child and Init}

One of the first tasks we had to handle was setting up the communication between init and child domains after spawning them. The following explains how we handled this during milestone 3.

We passed the init endpoint at a well known location in the childs task cnode when spawning the child. Afterwards we setup the communication between child and init by passing the childs endpoint to init and then sending a confirmation message from init to the child. The child only starts running from its main function after receiving this confirmation from init. The following describes these steps in more detail with some implementation details:

\begin{enumerate}
    \item When spawning a new child: Pass init endpoint in task cnode in slot defined by constant \verb|TASKCN_SLOT_INITEP| using \verb|cap_copy|. This is done in the function \verb|spawn_child_cspace_set_initep| in \verb|lib/spawn/spawn.c|.
    \item In \verb|create_child_channel| (spawn.c) register
    \verb|recv_setup_closure| from init
    (used to receive initial ep from child)
    \item in init.c (called before each thread):
        \begin{itemize}
            \item create child endpoint i.e channel for this child (we are already in child)
            \item register \verb|receive_init_closure| (used to save ep received from init)
            \item send child ep to init and wait until
            \verb|receive_init_closure| get called
        \end{itemize}
    \item \verb|recv_setup_closure| gets called:
        \begin{itemize} 
            \item save child ep in init channel struct (\verb|lmp_chan|)
            \item send message to child that channel is ready (in \verb|rpc_send_setup_closure|)
        \end{itemize}
    \item \verb|barrelfish_recv_init_closure| gets called on the child side
        \begin{itemize}
            \item child is now ready to communicate with init
            \item on child side continue after recv init success
            \item init is now dispatching events in \verb|recv_regular_closure|
        \end{itemize}
\end{enumerate}

\section{RPC Implementation}

In this section we detail how RPC over LMP was initially implemented. After setting up the communication between init and its children, mainly two files are concerned when talking about RPC at this stage of the project: \verb|lib/aos/aos_rpc.c| contains the implementation of all RPC functions that can be called by the child to communicate with a service. \verb|usr/init/rpc.c| contains the implementation of the services and their RPC interactions in init.

\subsection{Message Type}
When sending a message over LMP we can send 4 * 64 bytes of data and a capability. In our RPC implementation we reserve the first 64 bytes of each call for metadata. At this stage of the development we store the message type in the lowest 8 bytes of those 64 bytes. The message type is used to determine in init, which RPC call was made and in the child to check if the received response was for the correct RPC call.

In \verb|usr/init/rpc.c| there's a switch statement that is used to route the incoming RPC calls to the correct service function using this message type. The rest of the bytes are then interpreted by the service functions.

\subsection{init - Service Handler}
The init process was at this point dispatching waitset events at the end of the main function in \verb|usr/init/main.c|. After a communication with a child was setup, each time a child sent something to init over LMP it was received in \verb|usr/init/rpc.c| in the function \verb|rpc_handler_recv_closure|. That is also the where the aforementioned switch for the message type is located.
After selecting the correct service and handling the RPC request, a receive closure with the function \verb|rpc_handler_recv_closure| was registered again for the child channel. The service functions were designed to be non-blocking and could therefore only receive a single LMP message per call. This turned out to be too limiting as we wanted to receive more than that in some service functions.

\subsection{child - aos\_rpc}
The \verb|aos/aos_rpc.h| interface has a set of blocking functions that can be called by any child. Each child sets up the internal state of aos\_rpc and the channel communication with init before its main function gets called. There are multiple functions that return the RPC channel to the different services. All of those returned the init channel, as init handled all the RPC calls at this point.
In the main function a child could call any RPC function, which blocked and reported success or failure over the \verb|errval_t| return value. For example to send a character to the terminal service (running in init) the client could do the following:

\begin{lstlisting}[language=c, caption=Example of child calling an RPC function]
struct aos_rpc *serial = aos_rpc_get_serial_channel();
errval_t err = aos_rpc_serial_putchar(serial, 'c');
bool success = err_is_ok(err);
\end{lstlisting}

In the listing above the client gets the serial RPC channel (which in our case is pointing to init) and then sends the character 'c' over the blocking call \verb|aos_rpc_serial_putchar| and is able to tell if the transfer was successful by checking the returned \verb|errval_t| value.

\subsection{Blocking RPC}
Making the RPC calls blocking was not trivial and was deliberated quite a bit before comming to a conclusion on how we wanted to implement it. We knew that it was required to make the RPC calls blocking, but we had to still be able to dispatch waitset events, so that LMP reponses could actually be received while blocking. So we implemented a function \verb|aos_rpc_dispatch_until_set| in \verb|lib/aos/aos_rpc.c| which would take a pointer to a flag and then continuously dispatch on the waitset until this flag was set to true. Before calling this function we would register a closure to receive LMP messages from init on and set the bool flag to true when a message from init was received allowing \verb|aos_rpc_dispatch_until_set| to return.

\section{Realisations and Improvements}

When revising our work after finishing milestone 3 we realised that we had a lot of code duplication in the LMP channel handling. Having multiple functions of setting up a closure, registering receive closures, re-registering closures upon transient failure and so on only to set a flag seemed overcomplicated. So we wanted something to make it easier to write LMP code by abstracting away from the provided bare LMP channel functionality.

We also noticed that we experienced a lot of stack ripping as it was described in the book. Following the logic of the communication was made harder as for each message that was sent or received a closure has to be built and registered. The logic of a single RPC call was in init spread over several functions and hard to track.

In the child (in \verb|lib/aos/aos_rpc.c|) some effort was already made in that direction. There the RPC functions had to be blocking, which already required some abstraction. By providing the functions \verb|aos_rpc_lmp_send| and \verb|aos_rpc_lmp_call| as an interface for the RPC functions the LMP channel plumbing could be somewhat abstracted away.

Based on this idea a general LMP interface was developed, which hides all the underlying work with closure and provides a simple blocking interface which was then used by the child (in \verb|lib/aos/aos_rpc.c|) and in init (\verb|usr/init/rpc.c|). The following section describes this new abstraction layer in more detail:

\subsection{LMP Protocol}
\label{sec:lmp-protocol}
The abstraction was defined in the header (\verb|aos/lmp_protocol.h|) and implemented in the source file (\verb|lib/aos/lmp_protocol.c|). There functions for sending and receiving LMP messages were provided which had an interface like the following:
\begin{itemize}
    \item Simple send
    \begin{verbatim}errval_t lmp_protocol_send(struct lmp_chan *chan,
    uint16_t message_type, struct capref cap,
    uintptr_t arg1, uintptr_t arg2, uintptr_t arg3)\end{verbatim}
    \item Simple receive
    \begin{verbatim}errval_t lmp_protocol_recv(struct lmp_chan *chan,
    uint16_t message_type, struct capref *ret_cap,
    uintptr_t *ret_arg1, uintptr_t *ret_arg2, uintptr_t *ret_arg3)\end{verbatim}
    \item Send bytes
    \begin{verbatim}errval_t lmp_protocol_send_bytes_cap(
    struct lmp_chan *chan, uint16_t message_type, struct capref cap,
    size_t size, const uint8_t *bytes)\end{verbatim}
    \item Receive bytes
    \begin{verbatim}errval_t lmp_protocol_recv_bytes_cap(struct lmp_chan *chan,
    uint16_t message_type, struct capref *ret_cap,
    size_t *ret_size, uint8_t **ret_bytes)\end{verbatim}
\end{itemize}

All capabilities (cap) and numerical arguments (arg1, arg2, arg3) could be omitted and so further definitions were included to provide an interface for all possible combinations of arguments and capabilities. All these send and receive function in \verb|lmp_protocol| were blocking. Always waiting for the registered closure to be handled before returning. This blocking mechanism was implemented the same way as described above for "blocking RPC". The following snippet shows how this interface was used in the \verb|aos_rpc_process_spawn| function (code is simplified for demonstration):

\begin{lstlisting}[language=c, caption=Usage of aos/lmp\_protocol in aos\_rpc\_process\_spawn]
// Request process spawn
lmp_protocol_send0(&rpc->chan, AOS_RPC_PROCESS_SPAWN);
// Send commandline
lmp_protocol_send_string(&rpc->chan,
    AOS_RPC_PROCESS_SPAWN_CMD, cmdline);
// Get pid and success information
lmp_protocol_recv2(&rpc->chan,
    AOS_RPC_PROCESS_SPAWN, &ret_pid, &ret_success);
\end{lstlisting}

In the listing above we see a more complex communication than was previously implemented. We see that process spawning is initiated by sending a spawn request that consits only of the message type and 0 additional bytes. That is also what the 0 in \verb|lmp_protocol_send0| stands for. This is followed by sending a string containing the command line which should be used to spawn the process. At the end 2 integers are received which contain a success flag and if the spawning was successful the pid of the newly spawned process.

The code on the init side was adjusted accordingly to handle RPC calls with multiple LMP messages per call. The main entry point in \verb|usr/init/rpc.c| (function \verb|rpc_handler_recv_closure|) was left unchanged, but the service functions are now blocking functions, that also use the \verb|aos/lmp_protocol.h| interface.
For example the following listing shows an extract of the spawn process service function:

\begin{lstlisting}[language=c, caption=Usage of aos/lmp\_protocol in usr/init/rpc.c]
// Receive and parse commandline
lmp_protocol_recv_string(chan,
    AOS_RPC_PROCESS_SPAWN_CMD, &cmdline);
char **argv = make_argv(cmdline, &argc, &buf);
// Spawn process
init_spawn(argv[0], &pid);
// Send back pid and success flag
lmp_protocol_send2(chan,
    AOS_RPC_PROCESS_SPAWN, pid, true);
\end{lstlisting}

The listing above was again substentially shortened. Leaving out all of the error handling and the declarations. But it should show, that it is a close parallel to the \verb|lmp_protocol| calls made on the client side, just swaping out receive and send in the respective functions.
Noteworthy is, that there is no call to receive the first empty LMP message on the service side. This is because the first LMP message is always received by \verb|rpc_handler_recv_closure|. To prevent that all RPC calls had to be sent using 2 LMP messages, relevant bytes from this first LMP message are always passed on to the service function by \verb|rpc_handler_recv_closure|.

Another change introduced with the LMP protocol was, that now two bytes of the reserved 64 bytes of metadata per LMP message were used. Previously only one byte was used to send the message type. What changed is, that now one byte determines which RPC function the message belongs to and the other byte determines what kind of message is being sent. This allows each RPC call to have its own sub protocol for sending different kind of messages. For example in the listings above we can see that a different message type is used when sending the commandline.

\section{Bonus Objectives}

\subsection{Large Messages}
Sending large messages was implemented by sending consecutive LMP messages. We deliberated if we send frame capabilities and use shared memory to pass large messages. The advantage of sending large messages by sending frames would have been, that only one LMP message would have to be sent and therefore less context switches would be needed. Additional context switches are always taking away from the performance. The way LMP works at least 2 context switches per LMP message are needed when sending large messages using multiple LMP messages. One context switch for sending the LMP message and a context switch back to the sender so the next LMP message can be sent.

The disadvantage of using a frame is, that we considered it to be more complicated and time consuming to implement. Another disadvantage is that for messages that could be sent using a few LMP messages allocating a whole frame seemed to be a waste. It would be possible to create a system to reuse frames until the were filled up and then send new frames, but that would make the whole large message sending even more complicated.

The large message sending was integrated into \verb|lmp_protocol| too with the functions \verb|lmp_protocol_send_bytes_cap|, \verb|lmp_protocol_recv_bytes_cap|, \verb|lmp_protocol_send_string_cap| and \verb|lmp_protocol_recv_string_cap|. The send bytes function takes a pointer to a byte array and a size and sends in a first message the size in the fisrt 64 bytes and the initial 2 * 64 bytes in the remaining payload. Each of the following LMP messages contains 3 * 64 bytes until size bytes have been sent.

The receive bytes function works as a parallel to send bytes by allocating a buffer of the size that is sent with the initial message and then filling this buffer as bytes are being received over LMP messages.
The string sending and receiving works almost identical. The string send function takes a 0 terminated string and calculates the size of it and then proceeds to send the string by first sending the size and then all the bytes identically to how it was done for the send bytes function. Receiving the string works exactly the same as receiving the bytes, but the string receiving function does not return a size.

With both the bytes and the string functions for passing large messages there can also be sent a capability alongside. The capability is send with the first message that is passed. There is no guarantee that more than one message is passed, but always at least one message is passed. Therefore the interface allows to pass zero or one capability.

\section{Process Management}

TODO
(Does this belong here?)

\section{RPC Protocol}

TODO


\chapter{Page fault handling}

The use of virtual memory presents both benefits and drawbacks.
On one hand, it provides applications with a simple view of memory, enables relocation, isolates processes, improves performance of application startup via copy-on-write, and enables demand paging to disk.
On the other hand, it can reduce performance, as translations take time and an application does not get full control of its memory.

One use case of virtual memory is to allow an application to reserve a large region of memory, but use page faults to only allocate physical memory for those parts of it which are used (and only when they are used).
While in Unix page faults are handled by the kernel, in Barrelfish physical memory is managed by applications themselves (using capabilities), which means that page faults are handled by applications as well.
A self-paging application creates its own virtual-to-physical mappings, while the kernel handles unsafe operations and ensures safety (via capabilities) and completeness.

In this milestone we implemented page fault handling in applications.
We also re-designed paging and memory functionality to use more efficient data structures.
In the following sections we describe the steps we took, the design and challenges.


\section{Page faults}

As mentioned, a self-paging process manages its own page faults, by running a handler in userspace.
The handler maps a page when the application accesses it.
This requires an exception handling mechanism.


\subsection{Exception handling}

The first step was to enable exception handling.
A handler for a thread could be easily registered with \verb|thread_set_exception_handler|.
The exception handler is called from the same upcall that dispatches the application when exiting the kernel.
When it returns, the thread is restarted.

We had to set a separate exception stack for the exception handler.
This stack had to be allocated and mapped immediately whenever a thread was created, as otherwise we could get a recursive page fault, which is not supported.
The stack also needed to be per-thread, to prevent corruption.

This was enough to trigger a page fault whenever a non-mapped address was accessed.

\subsection{Mapping pages}

To map a page, the page fault handler needed to acquire a RAM capability and map it.
Acquiring the capability was done using \verb|frame_alloc|, which fetched it from the memory manager (over an RPC).
Mapping was done using the \verb|paging_map_fixed_attr| interface, mapping the new RAM capability to the virtual address of the page that the fault occurred in.

We also checked that a faulting address had indeed been allocated for the heap or stack (using e.g. \verb|paging_alloc|), by checking in our virtual address manager.
If it had not, the cause was a bug, and the process was aborted.

To make the reporting more useful, NULL pointer detection was also added.
Unfortunately there was not enough time to add a stack guard page, which would have helped to debug stack overflows.

Locking (with a mutex) was also added to the handler to ensure that two threads of a process would not map the same address at the same time.


\subsection{Heap management}

Page fault handling made it possible to lazily allocate memory for the heap.
The handout originally had a 16 MB static array as the heap.
We implemented the \verb|sys_morecore_alloc| function, which is called by \verb|malloc| when it runs out of memory.

Whenever \verb|malloc| called \verb|sys_morecore_alloc|, we reserved virtual memory using \verb|paging_alloc|, but did not map it.
Physical memory for the request was only allocated through page faults.
Instead of using \verb|paging_alloc|, a probably even faster alternative would have been to reserve a large virtual memory region, and return virtual addresses from it on future \newline \verb|sys_morecore_alloc| calls.


\section{Address space management}

In this milestone we also changed the data structures used by our paging code.
The page fault handler needed to look up virtual mappings and page tables based on virtual address.
Our existing implementation based on linked lists was too slow, so we reimplemented it using AVL trees.
In addition to managing the virtual address space and page tables with AVL trees, we also extended it to our memory manager (RAM capabilities), for further performance gains.

A detailed description of the design choices and resulting implementation can be found in Chapter 1.


\section{Testing}

To test our page fault handling, we wrote a test that \verb|malloc'ed| 64 MB of memory but only accessed a few bytes in the middle, and verified that it was fast.
We also tested that accessing many addresses and generating a large number of faults worked.
To test support for multiple threads, we allocated a memory region and had multiple threads access the same region.


\section{Bonus tasks}

We also implemented the bonus task of dynamic stack allocation.
We \verb|malloc'ed| the entire stack and only mapped it in (via page faults) when the stack grew to the next page.

Unfortunately, during further milestones we discovered some issues with this approach.
For example, if the stack happened to grow during a \verb|mutex_lock| operation, the domain would abort, as taking a page fault while the dispatcher is disabled was not allowed.
As we were not sure how to fix this situation, we later switched back to a static stack (mapping it in before a thread starts).




\chapter{Multicore}

Nowadays one can hardly find any single-core machine on the market, so
a modern operating system has to support multiple cores somehow. The
Toradex board has not only four Cortex-A35 cores in the main processor
cluster but also two Cortex-M4 processors which are used for real-time
processing and chip-management. A heterogeneous multi-core system like
the Toradex board poses entirely new challenges on operating systems
which were initially built to support machines with just one single-core
processor.

Luckily for us Barrelfish was designed for heterogeneous systems from the
ground up and solved many of the occurring problems in an elegant but also
radical way. Our task in this milestone was to bring up the second core
and do all the preparation so Barrelfish can simply power on the second core
and jump to an address we provided earlier. Then we had to setup communication
with the second core and finally split up the memory so we can use the existing
paging system on the other core.

\section{Bringing up a second core}

As starting a second core requires EL1 permissions, only the CPU driver can do
this task for us. This functionality was already implemented in the handout so
all we had to do was some bootstrapping and then call
\verb|invoke_monitor_spawn_core()| to let the kernel start the core. The whole
bootstrapping procedure was implemented in the provided \verb|coreboot()|
function in the file \verb|lib/aos/coreboot.c|. The function takes a core id, the
name of the boot driver, CPU driver and init image and a reference to the URPC frame.
The whole setup was similar to spawning a process and we could benefit from our previous
knowledge about multiboot images and ELF binaries.

First, we load the multiboot images into the current address space because we
need certain properties (e.g. size) about these images later on when allocating memory
for the other core. The function \verb|load_multiboot| takes the image name, loads the
image into a frame and maps the frame at a free address into memory so we can access it
later. The function returns a \verb|binary_info| struct which holds a reference to the
frame, the mapped address and the size of the binary. We call \verb|load_multiboot()|
with the image names provided as arguments to \verb|coreboot()| and get back three
\verb|binary_info| with information about the images.

Now we are ready to allocate memory for the new core. For the kernel control block (KCB)
we allocate a ram capability of size \verb|OBJSIZE_KCB| and retype it into a
\verb|ObjType_KernelControlBlock| capability. The comments say that the KCB needs 16k
alignment but in our case it also worked when the KCB is page aligned so that was maybe an
artifact left by the Pandaboard code.

The remaining memory can be allocated as simple frames.
We could allocate each frame individually or calculate the required size and allocate one
contiguous block of memory. As our paging system doesn't support arbitrary alignments we opted
for the later. Later we realized that a contiguous block was indeed the right decision because
it makes clearing the cache and other cleanups in the end much easier.

All the memory allocation happens in the \verb|allocate_memory()| function. The function takes
the size of all images and returns two structs. The \verb|core_mem_block| struct has all the
information about the contiguous block which we allocate as a large frame. The \verb|core_mem|
struct has an entry for each memory region required by the new core. Each memory region is
described by the \verb|mem_info| struct, which holds the size, a pointer to the location where
the region is mapped in the current address space and the physical address of the region. All these
values are required later, when we load/relocate the ELF binaries and fill in the core data struct.

In the \verb|allocate_memory()| function we first calculate the total size of the memory block and
the offset of each memory region within this block. We have to fulfill certain alignment requirements
but as we are not depended on the paging system, we can set arbitrary alignments and hence meet
every requirement. Once we laid out the address space, we know the size of the memory block and can
simply allocate a frame. Then we use \verb|frame_identify()| to get the physical address of the frame
and map it to our address space so we can write to it. The last step is to go through each memory region
and use the offset to set the pointer and the physical address of that region. It took some iterations
to get a clean design so that we could adjust the address layout in just one place, which avoids bugs
introduce by code duplication. Doing this extra effort payed out in the end during debugging, when the
second core didn't start and we could apply changes to the address layout in a safe way.

Now that we allocated all required memory we are almost ready to load the ELF binaries. As the physical
address of the entry point is calculated when we load a binary, we have to first find the virtual address
of the entry point by looking at the ELF symbols. The ELF library, shipped with the handout, has the
\verb|elf64_find_symbol_by_name()| function, which we use to find the virtual address of the entry point
from the boot driver and the CPU driver. Now we can use the provided function \verb|load_elf_binary()| to
load the boot driver and CPU driver. The boot driver runs with an 1:1 mapping of virtual to physical
addresses so no relocation is required. The CPU driver gets relocated to \verb|ARMv8_KERNEL_OFFSET| using
the \verb|reloacte_elf()| function.

At this point most of the bootstrapping is done but the other core has no clue where to find the binaries
and his memory regions. All these information is stored in the \verb|armv8_core_data| struct. Barrelfish
places a pointer to this struct into register \verb|x0| when starting the core. The new core can then look
up all the required information form this location.

We have already allocated memory for the core data struct, so we just have to fill in the required values.
This is pretty straightforward as the struct is well documented yet we picked the wrong address for the
CPU driver entry point. This mistake was hard to track down and we only discovered the error after enabling
the debugging functionality in the boot driver.

Before we can call \verb|invoke_monitor_spawn_core()| we have to consider ARMs weak memory model. All data
that we set up so far is likely to still sit in the cache of the first core. The second core cannot access
the cache of the first core so we have to bring the memory system to a so called \textit{Point of Coherency}.
The ARM architecture defines the point of coherency as the moment when all agents see the same thing for a
memory location at a given physical address. In our case we have to execute the \verb|DC CVAC| instruction.
We can do that by calling \verb|cpu_dcache_wb_range()| with a range of virtual addresses. As we allocated a
contiguous block before we can use the \verb|core_mem_block| struct to get the address range and clear the
correct part of the cache.

Now we have done all the required steps to start the second core. We simply call \verb|invoke_monitor_spawn_core()|
and the kernel will do the rest for us. As mentioned before we couldn't start the core after the first attempt
because we picked the wrong address for the CPU driver entry point but after some debugging we finally got the
second core up.


\chapter{User-level message passing}

In this milestone we had to setup a efficient communication channel between
the two cores and extend our RPC infrastructure with UMP, so we can send messages
between domains on different cores.

\section{UMP}

UMP is shared memory message passing construct.
From the users point of view, it provides one queue in each direction, e.g. each
user can send and receive.

For a receiver it's easy to know when a slot is ready to be read, he just needs
to wait until it's content is not zero anymore.
For sending this is a bit more complicated.
One could potentially also poll until the receiver has it cleared.
But this would lead to increased cache coherency traffic.
That's why acknowledgement messages are used to signal that slots are ready
again to be used for sending.

If data and acknowledgements would share a queue, that would constrain the
communication behaviour.
For example, acks might get stuck behind data messages, and a process might need
to send multiple messages back for each message it receives, leading to a
potential deadlock, if the receive side is not expensively compacted by moving
the ack messages in front of data messages.

That's way under the hood, For each queue the user sees, there two separate
queues.
One is for data messages, and one is for acknowledgement messages.
We look that acks don't induce a large overhead, that's way when acknowledging
data messages, we coalesce acks and send multiple acks with one acknowledgement
message.
When we send data, we potentially ack acks by using
the byte used to signal that the slot is ready to contain the slot to be acked.
As we have so little acknowledgement messages compared to data messages, there
won't be any imbalance from this.

\section{Inter-Core Communication Protocol}

In order to integrate UMP in our existing RPC infrastructure, we introduced a separate
library called AOS protocol. The library implements a similar interface as the LMP protocol
described in chapter \ref{sec:lmp-protocol}. The library is defined in the header 
\verb|aos/aos_protocol.h| and implemented in the file \verb|lib/aos/aos_protocol.c|. The semantics
of the functions are the same as in LMP protocol, but the message passing is transparent to the user
i.e. he doesn't know if he is talking to a service on the same or a remote core. A client can create
a channel using either \verb|make_aos_chan_lmp| or \verb|make_chan_ump|. This channel is then passed
to the specific function e.g. to send a message. The library decides then internally if the message
should be forwarded over LMP or if the UMP queue should be used.

For the \verb|init| dispatch loop, we used a new function called \verb|aos_protocol_wait_for()|. This
function uses \verb|event_dispatch_non_block()| to dispatch LMP messages and \verb|aos_ump_dequeue()|
to get messages over UMP. This function could potentially never dispatch LMP messages, if a lot of UMP
messages arrive over a channel, but this event seems unlikely.

The final step was replacing all LMP protocol function calls in \verb|aos_rpc.c| with functions from
AOS protocol. This enables inter-core communication between any service on the system.



\chapter{Nameserver}

\begin{itemize}
    \item implemented multi hop over init (URPC was not mature enough + is extra challenge)
    \item Init route messages based on did and channels (if msg comes from client chan it gets rerouted to server chan of other domain)
    \item every domain has two channels (only one handler per channel per domain)
    \item one channel is used if domain acts as client (can block)
    \item the other channel is used if domain acts as server (handler was registered)  
\end{itemize}


\chapter{Network}

\section{Architecture / Design}

\subsection{Running the Network Driver}

One of the earliest decision in the network project was, where the driver should run. One of the considered options was, to start an additional core that would be dedicated to run the network driver and handle the network protocols. The advantages of this approach would be:
\begin{itemize}
    \item Better performance and lower latency: The network could run uninterrupted and be preemted much less and only for the kernel. (This is an assumption, as we did not try to implement this approach and meassure it.)
    \item Fast client applications that use networking: Because the networking would run on a different core than any other application, UMP could be used as the sole message passing system. This would not only make the interface simpler but also faster, as the book has already shown that UMP is faster than LMP.
\end{itemize}
The disadvanteges would be:
\begin{itemize}
    \item Refactoring multi-core memory management: As detailed in previous chapters, we split the memory between the two already running cores. We agreed as a team, that it would not make much sense to split the memory threeways for the network and it is hard to predict which core would need how much memory. So for a networking core to be functional we would have to implement a way to pass memory to the netwroking core, which we previously decided not to do because we deemed it to be a lot of complicated work.
    \item Unrealistic: If we were to continue work on this OS we would at some point want to start all the cores and use them for user applications. At this point we would not want a core dedicated to networking anymore. When looking at the bigger picture, it therefore does not seem to make sense to have a dedicated networking core and would also feel like cheating to be able to show better performance in the final report.
\end{itemize}

Considering the arguments above we decided to run the networking driver in its own domain which is pinned to core 0.

The idea of running the driver in the main process was quickly discarded as we already consider that to be a too big monolith that we would like to trim down. Also the idea of running the driver on multiple cores sounds like a nightmare and was not further considered.

\subsection{Running the Network Protocols}

The next decision was where to run the code for the different network protocols and how they would interact with the driver and each other. The book already suggests adding the protocol specific code to the driver application. This comes at a cost of flexibility and modularity, but is way simpler and faster to implement. Additionaly the nameservice project was not started at this point, so the communication between protocols would have to be improvised and be rewritten later on.

With those arguments in mind, the decision to implement the protocols inside the driver application. This also allowed to defer the implementation of the communication with other applications until nameservice made some progress, which was in retrospect a great time saver.

\subsection{General Design Decisions}

One of the design goals in the networking project was to reduce copying of data and therefore passing a reference into the same ethernet frame between the network protocol. Another goal was to keep the number of mallocs low and allocate on the stack whenever possible. This should simplify resource management and reduce the amount of bugs.

An important design decison was to keep the interface between the network protocols simple, straight forward and without a lot of abstraction. This makes the network stack less extensible and violates the open-closed principle when e.g. adding a new protocol. The advantages are, that the its faster to implement like this and there is no overhead caused from the abstraction.

That beeing said, we made it a goal to separate the code for the different protocols: Each network protocol has its own files and a clearly defined interface for interacting with other protocols. For example the IP protocol has its own header and source file and all the IP code is inside those files. When calling functions to send IP packages, the code of the IP protocol will call functions of the ethernet implementation to send IP packages over ethernet frames.

\section{Driver}

The driver to interact with the networking hardware was given in the handout. Because its functionality was not changed, the driver itself will not be covered in the report but only the interaction with it.

At startup the driver is handed two large frames: One to write incoming ethernet frames to and the other to read outgoing ethernet frames from. The driver uses slices that are half a page each per ethernet frame. So that a page is used for up to two ethernet frames.

To interact with the driver enqueue and dequeue operations are used. To get access to received ethernet frames the dequeue operation is used. It gives an offset into one of the memory regions used and blocks that part of the memory until it is released back to the driver. To do so the enqueue operation is used. Transmiting outgoing ethernet frames works in a similar way. The client of the driver has to keep track on which frames can be written to. Ethernet frames can be handed to the driver with an enqueue operation. To get back access to the memory after the sending is done, the dequeue operation has to be called again.

Because the enqueue and dequeue operations do not give a virtual address or pointer to work on, but only an offset to the base of the large frames passed to it, we decided to map those frames again a second time for use in the protocols. This we operate on different virtual addresses for the same physical memory in the same domain. The advantage of this was, that the queue interface used by the driver had not to be changed or rewritten.

\subsection{Disabled Caching}
A big problem early on was, that parts of the sent ethernet frames were sometimes not correct. On closer investigation the wrong parts were identical to earlier ethernet frames (e.g. sending an ICMP echo reply contained bytes from an earlier sent ARP request). After suspecting a caching problem, the memory used to interact with the driver was mapped with deactivated caches and this solved all of these problems. Later on we discovered that the same was suggested on the moodle forum to other groups, so we did not search for an alternative. But we suspect that this is currently the main bottleneck in the networking code.

\subsection{Copy of Data}
Earlier we said that a design goal was to keep the copying of data to a minimum. But after seeing how the ethernet driver worked we decided to not share to which the driver wrote. The main reason for this was a security concern: The minimal amount of memory that can be shared is one page. But one page holds two ethernet frames. Because two ethernet frames sharing a page might not belong to the same client it would be unsafe to share the memory, as then a client could get access to data that belongs to another client. This could even be forced by producing a lot of traffic.

It would be maybe possible to rewrite the driver to handle it differently, so that the memory could safely be shared with clients. But we deemed this to be to big of a task for the limited time available.

So we had to add at least one copy to send the data to the client. But we made sure it was minimal, by only copying the payload, but none of the protocol headers. Beacause we wanted to send some additional data like the ip address and port numbers, those were copied to the back of the memory, right after the payload. This can be safely done, because the memory reserved per ethernet frame by the driver is substentally bigger than an ethernet frame could be. The driver does this, so the ethernet frames are aligned to half the page size.

\section{Ethernet}
The ethernet handling was implemented in \verb|usr/drivers/enet/ethernet.c| and its interface defined in \verb|usr/drivers/enet/ethernet.h|.

\subsection{Receiving Frames}
Receiving frames was rather straight forward. The main event loop in \verb|usr/drivers/enet/enet_module.c| is constantly checking for incoming ethernet frames. When receiving an ethernet frame (over driver dequeue), it calls the ethernet handle function which is blocking. As soon as this function returns the memory slice in which the incoming ethernet frames resided in was released again by calling enqueue on the driver.

The handle function would look at the ethernet frame and call the IP or ARP handle functions respectively. The ethernet frame would be discarded and not handled if the protocol was unknown or the destination ethernet address was invalid. The address would be considered invalid if it was not a broadcast address or the ethernet address of the Toradex board.

\subsection{Sending Frames}
The main difficulty for this part was managing the memory regions for sending ethernet frames correctly and efficiently. We decided to create a linked list with nodes for each ethernet frame sized memory slice. If the node was in the linked list, the slice was not currently in use. Because we already know the amount of slices in advance the nodes could be allocated all at once and the linked list created when initialising ethernet.

What makes this approach performant is, that the list has never to be traversed but only ever the first element has to be touched in each operation: For reserving a slice that can later on be sent, the front element of the list has to be removed and its address remembered. To release the reserved slice again the remembered element had to be simply added back to the front of the list.

\begin{figure}
    \centering

    \begin{sequencediagram}
      \newthread{cli}{client}
      \newinst{eth}{ethernet}
      \newinst{drv}{driver}
  
      \begin{call}{cli}{start\_send()}{eth}{id, ptr}
        \begin{sdblock}{loop}{}
            \begin{call}{eth}{dequeue}{drv}{}
            \end{call}
        \end{sdblock}
      \end{call}

      \begin{call}{cli}{write\_payload(ptr)}{cli}{}
      \end{call}

      \begin{call}{cli}{send(id)}{eth}{}
        \begin{call}{eth}{enqueue()}{drv}{}
        \end{call}
      \end{call}
    \end{sequencediagram}

    \caption{Sequence diagram for sending ethernet frames}
    \label{fig:ethsend}
\end{figure}


The interface to send ethernet frames was built on that approach. Figure \ref{fig:ethsend} shows a sequence diagram of said interface. It consits of a function to start sending an ethernet frame, which returns an id and a pointer. The pointer can then be used to write the payload of the ethernet frame (e.g. an ip package). When done writing, the client calls the send function with the id. This id can directly be used by the ethernet protocol to enqueue the correct memory slice to the driver.

Another important part is, that the memory is not immediately free to be used again, because the data has to be sent by the hardware before that. This is done in parallel and it would be a waste of performance to block on that. So we decided to keep the data reserved and return from the send function. How data is released again can be seen in the same sequence diagram \ref{fig:ethsend}. In the start send function, ethernet dequeues all slices that were done sending from the driver. To figure out which list nodes to add in the tracking of free memory slices we use some pointer arithmetic. By using the fact that all of the list nodes were allocated as one big chunk and the address of each list node therefore correspondes to the address of a memory slice, the address of the correct list node can be quickly calculated using the offset returned by dequeue.

\section{ARP - Address Resolution Protocol}
The ethernet handling was implemented in \verb|usr/drivers/enet/arp.c| and its interface defined in \verb|usr/drivers/enet/arp.h|.


\section{IP - Internet Protocol}

\section{ICMP - Internet Control Message Protocol}

\section{UDP - User Datagram Protocol}

\subsection{UDP Checksum}

To calculate the UDP checksum we extended the checksum functionality in \verb|lib/netutil/checksum.c| with the function \verb|inet_checksum_IP_pseudo| which also takes the pointer to the start of the IP header. The UDP checksum requires additionaly to the UDP header and payload also that a pseudo header containing source IP, destination IP, protocol (which should always be 17 for UDP in this case) and the length of UDP datagram (length of UDP header + length of UDP payload in bytes).
Because of how the IP package and UDP datagram are already written directly to the reserved network buffer to reduce the number of memory copies, the pseudo header is never actually constructed. Instead, a pointer to the IP header is passed to the checksum function along with the pointer to the UDP header and the length of the UDP datagram. Using the IP header, the UDP checksum can be calculated by taking the IP addresses and IP protocol directly from the IP header.
The checksum function was designed to be also useable for calculating the TCP checksum. But this was not tested, as TCP was not implemented in the end due to time restrictions and other functionality having priority.

\section{Interface / API}

\section{Nameservice Interaction}

\chapter{Filesystem}

To stand up the filesystem one needs to call \verb|filesystem_init|, and when
exiting \verb|filesystem_unmount|, to flush all cached data to the sdcard.

\section{SDCard}

To initialise the SDCard, we first need it's device frame.
We can get that either by directly refering to the device frame capability put
into our cspace on startup, which works when we run as init.
Or we use an rpc call (\verb|aos_rpc_get_device_cap|) to request the device
frame from init.

Then we simply call \verb|sdhc_init|, as laid out in the book.

\section{VFS}

We don't remember why we implement vfs as a first step, maybe handing the fat32
methods directly to glibc wasn't obvious enough.

For vfs to work, we need to introduce concepts so that dispatching code can
easily forward the invocations to the right implementation.
These are to be found at \verb|lib/fs/fs_internal.h|.
We have \verb|fs_mount|, which bundles the common function pointers, and an
opaque state.
When forwarding an invocation, one just prepends the opaque state to the
argmuments, and passes them to the matching function pointer.
Additionally, we need a \verb|fs_handle|, which capsulate the state of an open
file/directory.
It has a member an \verb|fs_mount|.
So when dispatching for an open file/directory, on just prepends first the state
of the file, then prepends the state of the mount, passes them to the matching
function pointer.
This make the dispatching code (\verb|lib/fs/fopen.c|,
\verb|lib/fs/dirent.c|) fairly regular.

For FAT32 to be supported, we need to expand ramfs to understand not only file
and directories, but also mountpoints. If during path resolution it encounters a
mountpoint, the call dispatched as would the central dispatcher do.

In theory, one could initialise the glibc dispatchers (\verb|lib/fs/fs.c|, call
to \verb|fs_libc_init|) directly with the fat32 filesystem, but we didn't test
that constellation.

\section{FAT32}

Generally, the FAT32 code tries to follow the spec to the letter.
Also, the functions mirror the behaviour of the ramfs functions (with common
sense), as they are not otherwise specified.

The FAT32 filesystem is made out of 4 things, the metadata in the
\verb|BIOS parameter block| (BPB), the cluster chains in the fat, the
directories, and the normal files.
To note is that directories are basically just special files.

We deparse the BPB in \verb|lib/fs/fat32fs_internal.c:fs_read_metadata|
(a file that also contains all the cruft from the first trials).
This deparsing is specific to FAT32.

--- END ---

Implement a FAT32 Filesystem, spawn a process with the binary loaded off of the
sdcard, create fileserver to expose filesystem functionality over rpc.

Steps taken in implementation:
\begin{itemize}
	\item Added sdcard initialisation
	\item Methods to read/write from/to sdcard, with cache invalidation/writeback
	\item Parsing FAT32 metadata
	\item Tried basic listing of directory
	\item Added virtual filesystem capability
	\item Added capability to ramfs to mount other filesystems
	\item Added rpc call to get device frame
	\item Implemented FAT32 filesystem according specification
	\begin{itemize}
		\item Enable opening, listing, and closing of directories
		\item Enable opening, reading, and closing of files
		\item Enable seeking, and tell for files
		\item Enable creation, deletion of directories
		\item Enable creation, deletion of files
		\item Enable stat, but can't be used because don't get dirhandle for files
					from anywhere
	\end{itemize}
	\item Added tests to /usr/test/filereader/main.c
	\item Tested if elf image from sdcard boots
	\item TODO: Expose interface over rpc, hook int /lib/fs/fopen.c, add fd and
				hook into /lib/fs/dirent.c to call rpc if not fileserver
\end{itemize}

Specifics of virtual filesystem:
\begin{itemize}
	\item Describe datastructures of virtual filessystem
	\begin{itemize}
		\item fs\_mount
		\item fs\_handle
		\item having fs\_mount avaiable all the time
	\end{itemize}
	\item Describe path resolution (and whish to add layer external to specific
				fs to deduplicate resolution logic, cache results)
	\item Describe call dispatching in virtual filesystem
\end{itemize}

Specifics of fat32:
\begin{itemize}
	\item Describe datastructures specific to FAT32
	\begin{itemize}
		\item fat32\_fs
		\item fat32\_dirent
		\item fat32\_dirstate
	\end{itemize}
	\item Describe separate buffers for fat metadata and data
	\item Describe lazy write mechanism of sectors
	\item Idea to have separate buffers per filehandle/dirhandle, improve locality
\end{itemize}

\section{Limitations}

Needing to zero initialise a full cluster everytime a directory is created,
which is unnecessary but specified as such.
One could just keep the entry after the last zero, and if the cluster is full,
just look if our fat entry is eof.

Generally the filesystem is quite slow.
This can be attributed to data not being aggresively enough cached.
Especially the path resolution should be factored out of the specific
filesystems and aggresively cache results.
Filesystem wouldn't then not need to bother anymore with mounts and delegating
invocations, except for handling the mounts as a special file type, so as to
signal this to the path resolution, which then would do the delegation.
Also, the sdhc driver is not optimised, so far as we understand, making reading
a binary off of the sdcard very slow.

Seek of FAT32 doesn't allow for expansion of files, as FAT32 doesn't support
file holes.

Except for that, the filesystem should work correctly, even when there are no
more ressources (only sdcard, we exclude ram etc.) to serve, it should continue
to report error conditions accordingly, but not thrash itself.


\chapter{Shell}

TODO



\chapter{Summary of overall system}

It proved to have been good decision to invest time in writing test code
(see \verb|lib/grading|, but also other places like \verb|usr/memeater/main.c|,
\verb|usr/test/filereader/main.c|).
Because during the intergration we did not encounter further problems with the
core system functionality like paging and memory allocation.

We did not end up with as a modular system as imagined in the beginning, with
everything being its own service in its own domain.
It's quite tempting and also prudent given the little time that if something
works like the memory server, to not refactor it into it's own domain.

So init still does all of the core os work, and only the network project managed
to be truly in it's separate domain.

Especially in the rpc part, it felt sometimes more like programming for a
distributed system than just simply for a multicore processor.
All the many things that happen on Linux hiddenly are much more in present, and
that's a good thing.
One much more concerns one self with the reality and limitations of the system
this way.
Which is good, because it allows one solve the problem in a way that's
appropriate to the problem, not having to rely on that the abstractions have
enough holes in them to not stand in the way.

Even though it operates on radically different principles, because the
principles are so elegant, simple and more importantly, appropriate to todays
problems, one feels very much at home even after this short a time.

\appendix

\chapter{User guide}

\section{Shell}

To run the shell, the \verb|INIT_EXECUTE_SHELL| macro in \verb|usr/init/main.c| must be set to 1.

If an SD card is attached to the board, then the \verb|SDCARD_PRESENT| macro in \verb|usr/shell/main.c| can be set to 1.
If an SD card is not attached, then the macro must be set to 0.

\section{Networking}

To enable networking, the \verb|INIT_EXECUTE_ENET| macro in \verb|usr/init/main.c| must be set to 1.

\section{Filesystem}

Linux has the tendency to create vfat names if filename is not all uppercase.
This is especially important as the filesystem not properly recognise that it's
a vfat name and chokes on the zero byte terminator that the vfat name
necessarily contains.

To spawn a process off of the sdcard, one needs to first copy the binary onto
the sdcard, to make sure the binary is up to date, compatible with the current
code state.
The one needs to change \verb|INIT_EXECUTE_HELLO_SDCARD| macro in
\verb|usr/init/main.c| must be set to 1.
Also, the shell should be disabled for this (\verb|INIT_EXECUTE_SHELL|), as else
there might be two initialisations of the filesystem.
We used the hello binary, if a different binary is to be spawned, the path
under the should be changed appropriately.


\printbibliography

\end{document}
