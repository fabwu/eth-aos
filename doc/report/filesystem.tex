\chapter{Filesystem}

To stand up the filesystem one needs to call \verb|filesystem_init|, and when
exiting \verb|filesystem_unmount|, to flush all cached data to the sdcard.

\section{SDCard}

To initialise the SDCard, we first need it's device frame.
We can get that either by directly referring to the device frame capability put
into our cspace on startup, which works when we run as init.
Or we use an rpc call (\verb|aos_rpc_get_device_cap|) to request the device
frame from init.

Then we simply call \verb|sdhc_init|, as laid out in the book.

\section{VFS}

We don't remember why we implement vfs as a first step, maybe handing the fat32
methods directly to glibc wasn't obvious enough.

For vfs to work, we need to introduce concepts so that dispatching code can
easily forward the invocations to the right implementation.
These are to be found at \verb|lib/fs/fs_internal.h|.
We have \verb|fs_mount|, which bundles the common function pointers, and an
opaque state.
When forwarding an invocation, one just prepends the opaque state to the
arguments, and passes them to the matching function pointer.
Additionally, we need a \verb|fs_handle|, which encapsulate the state of an open
file/directory.
It has a member an \verb|fs_mount|.
So when dispatching for an open file/directory, on just prepends first the state
of the file, then prepends the state of the mount, passes them to the matching
function pointer.
This make the dispatching code (\verb|lib/fs/fopen.c|,
\verb|lib/fs/dirent.c|) fairly regular.

For FAT32 to be supported, we need to expand ramfs to understand not only file
and directories, but also mountpoints. If during path resolution it encounters a
mountpoint, the call dispatched as would the central dispatcher do.

In theory, one could initialise the glibc dispatchers (\verb|lib/fs/fs.c|, call
to \verb|fs_libc_init|) directly with the fat32 filesystem, but we didn't test
that constellation.

\section{FAT32}

Generally, the FAT32 code tries to follow the spec to the letter.
Also, the functions mirror the behaviour of the ramfs functions (with common
sense), as they are not otherwise specified.

The FAT32 filesystem is made out of 4 things, the metadata in the
\verb|BIOS parameter block| (BPB), the cluster chains in the fat, the
directories, and the normal files.
To note is that directories are basically just special files.

We deparse the BPB in \verb|lib/fs/fat32fs_internal.c:fs_read_metadata|
(a file that also contains all the cruft from the first trials).
This deparsing is specific to FAT32.

\subsection{Datastructures}

For the metadata and sd blocks, there is the \verb|struct fat32_fs|.
It also holds the sdhc driver state.

There are two sd blocks, one for the fat datastructures, and one for the data
itself.
This works, because all the methods concerning these area of data don't overlap,
so that was easy to introduce.
Data is written back lazily when for the same block a different sector is read.
This helps in reducing block transfer when for example manipulating the cluster
chains in the fat.
For that the blocks have a field for the sector their buffer currently contains,
and if the data is dirty.

To represent an open file, we have the \verb|struct fat32fs_dirent|
datastructure.
It stores the name of the file, it's size if it's a regular file, it's start
cluster.
Additionally, stores the position of its dir entry, which allows for fast
removal and updating of dir entry, without needing to reiterate through the dir
again to find the file.

It also holds something called the \verb|fat32fs_dir_state|, which is a bit a
misnomer because it also used for reading and writing of files.
It tracks the position in the file/dir.
\begin{itemize}
	\item \verb|clus| together with \verb|depth| and \verb|is_eof| keep track of
				which cluster we are currently at.
				Cluster is special, in that it's a single linked list, so we can't go
				back, and it's expensive to got to a specific element, and also just
				from knowing the cluster number we don't know how many elements in we
				are.
				\verb|depth| so we now at which offset from the beginning we are.
				\verb|is_eof| should have been used to speed up
				\verb|fat32fs_is_last_dir_entry|, but that didn't seem to happen.
				\verb|fat32fs_is_last_dir_entry| is there because it would be expensive
				if we overwrote \verb|clus| with the eof mark, and then to expand or do
				anything again, having to iterate through the whole cluster chain again.
	\item \verb|sector| is just the sector number relative to the cluster.
	\item \verb|des_pos| is specific to files, it tracks the offset in the file
				in bytes from 0, the beginning.
				It does not necessarily need to be in sync with \verb|clus|, \verb|sec|,
				as it just indicates the position we wish to read from, write to, e.g.
				set by a \verb|fseek| call.
	\item \verb|entry| is specific to directories, it tracks the offset in bytes
				from the current sector, and is incremented in 32 byte steps, the size
				of a directory entry.
\end{itemize}

% TODO: More content ideas
% --- END ---
% 
% Implement a FAT32 Filesystem, spawn a process with the binary loaded off of the
% sdcard, create fileserver to expose filesystem functionality over rpc.
% 
% Steps taken in implementation:
% \begin{itemize}
% 	\item Added sdcard initialisation
% 	\item Methods to read/write from/to sdcard, with cache invalidation/writeback
% 	\item Parsing FAT32 metadata
% 	\item Tried basic listing of directory
% 	\item Added virtual filesystem capability
% 	\item Added capability to ramfs to mount other filesystems
% 	\item Added rpc call to get device frame
% 	\item Implemented FAT32 filesystem according specification
% 	\begin{itemize}
% 		\item Enable opening, listing, and closing of directories
% 		\item Enable opening, reading, and closing of files
% 		\item Enable seeking, and tell for files
% 		\item Enable creation, deletion of directories
% 		\item Enable creation, deletion of files
% 		\item Enable stat, but can't be used because don't get dirhandle for files
% 					from anywhere
% 	\end{itemize}
% 	\item Added tests to /usr/test/filereader/main.c
% 	\item Tested if elf image from sdcard boots
% 	\item TODO: Expose interface over rpc, hook int /lib/fs/fopen.c, add fd and
% 				hook into /lib/fs/dirent.c to call rpc if not fileserver
% \end{itemize}
% 
% Specifics of virtual filesystem:
% \begin{itemize}
% 	\item Describe datastructures of virtual filessystem
% 	\begin{itemize}
% 		\item fs\_mount
% 		\item fs\_handle
% 		\item having fs\_mount avaiable all the time
% 	\end{itemize}
% 	\item Describe path resolution (and whish to add layer external to specific
% 				fs to deduplicate resolution logic, cache results)
% 	\item Describe call dispatching in virtual filesystem
% \end{itemize}
% 
% Specifics of fat32:
% \begin{itemize}
% 	\item Describe datastructures specific to FAT32
% 	\begin{itemize}
% 		\item fat32\_fs
% 		\item fat32\_dirent
% 		\item fat32\_dirstate
% 	\end{itemize}
% 	\item Describe separate buffers for fat metadata and data
% 	\item Describe lazy write mechanism of sectors
% 	\item Idea to have separate buffers per filehandle/dirhandle, improve locality
% \end{itemize}

\section{Limitations}

\verb|filesystem_init| can be called from any domain, so technically any domain
can use the filesystem. But we do not have a central fileserver yet, which
exposes the api via rpc. For \verb|lib/fs/fopen.c|, this would haven been easy,
because we already have the fd number as an easy to transmit identifier, so it
would have just been a one to one call over rpc.
For \verb|lib/fs/dirent.c|, we would have needed to add the fdtab structure, and
then we could have proceeded the same way.
One optional thing could be for the fileserver to make it more secure is add an
intermediate translation table.
This way the client can't infer from the gaps in the fd what other clients are
doing.
More importantly, this would prevent clients from spoofing their fd, e.g. they
would only able to use their own file descriptors that are in the translation
table, others wouldn't translate.

We don't honour if the fat is mirrored, e.g. we update at most one fat.

Needing to zero initialise a full cluster every time a directory is created,
which is unnecessary but specified as such.
One could just keep the entry after the last zero, and if the cluster is full,
just look if our fat entry is eof.

If staying with the zero initialisation, it might have made sense to have a zero
buffer and directly write it, instead of zeroing out one sector after the other
(reading, then zeroing).
Zeroing a cluster is also a perfect case for having dma requests of multiple
sectors, which then could also be used for the normal reading and writing.

Generally the filesystem is quite slow.

This can be attributed to data not being aggressively enough cached.
There should be at least 3 sd blocks per file, more if one introduces writing of
multiple blocks in one go.
The 3 blocks would be for the fat cluster chain, the dir entry, and finally the
content of the file.
This would alleviate the problem that while writing, the current data block is
contended for updating the file size and file content, which are different
sector, leading to unnecessary reads and writes of blocks.
To not have concistency problems, there could be hash table, where the sd blocks
are indexed by sector number, and also have a ref counter.
This would still allow "separate" sd blocks per file, but all would be
automatically consistent with other.
Also unmount would be made very, just flush the hash table.

ALso the path resolution should be factored out of the specific filesystems and
aggressively cache results.
Filesystem wouldn't then not need to bother anymore with mounts and delegating
invocations, except for handling the mounts as a special file type, so as to
signal this to the path resolution, which then would do the delegation.

Also, the sdhc driver is not optimised, so far as we understand, making reading
a binary off of the sdcard very slow.

Seek of FAT32 doesn't allow for expansion of files, as FAT32 doesn't support
file holes.

Except for that, the filesystem should work correctly, even when there are no
more resources (only sdcard, we exclude ram etc.) to serve, it should continue
to report error conditions accordingly, but not thrash itself.
