\chapter{Filesystem}

Implement a FAT32 Filesystem, spawn a process with the binary loaded off of the sdcard, create fileserver to expose filesystem functionality over rpc.

Steps taken in implementation:
\begin{itemize}
	\item Added sdcard initialisation
	\item Methods to read/write from/to sdcard, with cache invalidation/writeback
	\item Parsing FAT32 metadata
	\item Tried basic listing of directory
	\item Added virtual filesystem capability
	\item Added capability to ramfs to mount other filesystems
	\item Added rpc call to get device frame
	\item Implemented FAT32 filesystem according specification
	\begin{itemize}
		\item Enable opening, listing, and closing of directories
		\item Enable opening, reading, and closing of files
		\item Enable seeking, and tell for files
		\item Enable creation, deletion of directories
		\item Enable creation, deletion of files
		\item Enable stat, but can't be used because don't get dirhandle for files from anywhere
	\end{itemize}
	\item Added tests to /usr/test/filereader/main.c
	\item Tested if elf image from sdcard boots
	\item TODO: Expose interface over rpc, hook int /lib/fs/fopen.c, add fd and hook into /lib/fs/dirent.c to call rpc if not fileserver
\end{itemize}

Specifics of virtual filesystem:
\begin{itemize}
	\item Describe datastructures of virtual filessystem
	\begin{itemize}
		\item fs\_mount
		\item fs\_handle
		\item having fs\_mount avaiable all the time
	\end{itemize}
	\item Describe path resolution (and whish to add layer external to specific fs to deduplicate resolution logic, cache results)
	\item Describe call dispatching in virtual filesystem
\end{itemize}

Specifics of fat32:
\begin{itemize}
	\item Describe datastructures specific to FAT32
	\begin{itemize}
		\item fat32\_fs
		\item fat32\_dirent
		\item fat32\_dirstate
	\end{itemize}
	\item Describe separate buffers for fat metadata and data
	\item Describe lazy write mechanism of sectors
	\item Idea to have separate buffers per filehandle/dirhandle, improve locality
\end{itemize}
