\chapter{User-level message passing}

In this milestone we had to setup a efficient communication channel between
the two cores and extend our RPC infrastructure with UMP, so we can send messages
between domains on different cores.

\section{UMP}

UMP is shared memory message passing construct.
From the users point of view, it provides one queue in each direction, e.g. each
user can send and receive.

For a receiver it's easy to know when a slot is ready to be read, he just needs
to wait until it's content is not zero anymore.
For sending this is a bit more complicated.
One could potentially also poll until the receiver has it cleared.
But this would lead to increased cache coherency traffic.
That's why acknowledgement messages are used to signal that slots are ready
again to be used for sending.

If data and acknowledgements would share a queue, that would constrain the
communication behaviour.
For example, acks might get stuck behind data messages, and a process might need
to send multiple messages back for each message it receives, leading to a
potential deadlock, if the receive side is not expensively compacted by moving
the ack messages in front of data messages.

That's way under the hood, For each queue the user sees, there two separate
queues.
One is for data messages, and one is for acknowledgement messages.
We look that acks don't induce a large overhead, that's way when acknowledging
data messages, we coalesce acks and send multiple acks with one acknowledgement
message.
When we send data, we potentially ack acks by using
the byte used to signal that the slot is ready to contain the slot to be acked.
As we have so little acknowledgement messages compared to data messages, there
won't be any imbalance from this.

\section{Inter-Core Communication Protocol}

In order to integrate UMP in our existing RPC infrastructure, we introduced a separate
library called AOS protocol. The library implements a similar interface as the LMP protocol
described in chapter \ref{sec:lmp-protocol}. The library is defined in the header 
\verb|aos/aos_protocol.h| and implemented in the file \verb|lib/aos/aos_protocol.c|. The semantics
of the functions are the same as in LMP protocol, but the message passing is transparent to the user
i.e. he doesn't know if he is talking to a service on the same or a remote core. A client can create
a channel using either \verb|make_aos_chan_lmp| or \verb|make_chan_ump|. This channel is then passed
to the specific function e.g. to send a message. The library decides then internally if the message
should be forwarded over LMP or if the UMP queue should be used.

For the \verb|init| dispatch loop, we used a new function called \verb|aos_protocol_wait_for()|. This
function uses \verb|event_dispatch_non_block()| to dispatch LMP messages and \verb|aos_ump_dequeue()|
to get messages over UMP. This function could potentially never dispatch LMP messages, if a lot of UMP
messages arrive over a channel, but this event seems unlikely.

The final step was replacing all LMP protocol function calls in \verb|aos_rpc.c| with functions from
AOS protocol. This enables inter-core communication between any service on the system.

