\chapter{Memory management and capabilities}

123 \cite{aos-book}

The memory manager manages ranges of physical address space, we started to call
them memory ranges.
If one compares with paging, there there is a manager of ranges of virtual
address space, called addr\_mgr.
What both have in common is the invariant that ranges are to be disjunct, and
that a certain range should only be given out once, from the managers view there
exists only one owner of a certain range.
That's why both have quite similiar code, and one might have found a common
abstraction to refactor out, get the memory manager to use the addr\_mgr.
But the memory manager must also cope with the fact that there exists a physical
representation of it's ranges, the ram capabilites. And it needs to split them
according to a given request.

Ram capabilities are the foundation on which everything else (except device
frame), is built upon in the barrelfish world.
One needs a ram cap to create cnodes to store capabilites.
One needs ram caps to store the shadow paging tables.
One needs ram caps to for allocators.

\section{Access Control}

From an access control point of view, the memory manager does the following thins:
\begin{itemize}
	\item Propagation, in that it copies capabilities to send over lmp
	\item Restriciton, it splits the capabilities received from the boot info
	\item Amplification, ordinary processes don't have the boot info capabilities
		 		use RPC to get ram caps
\end{itemize}

\section{How}

The init process receives via the second argument (\verb|argv[1]|) the bootinfo
structure.
This structure contains information about the memory regions at \verb|regions|.

We iterate through this array of memory regions, and search for a memory region
of type \verb|RegionType_Empty| (which signifies its a ram cap).
We pass this to \verb|usr/init/mem_alloc.c:initialize_ram_alloc|.

This initialises the memory manager (\verb|lib/mm/mm.c:mm_init|).
An overview over the datastructures initialised is given in \ref{mem-data}.
We need to initialiase the three slab allocator used in the memory manager.
Also it zero initalises the member of \verb|struct mm|, so that we would catch
errors early.
Additionally, we initialise a mutex (\verb|alloc_mutex|), which is needed later
on when guarding against concurrency bugs.

After that, we grow each slab with statically allocated arrays (with space for
32 \verb|bi_node|'s, 64 \verb|mm_node|'s, and 64 \verb|aos_avl_node|'s).
This should be enough to satisfy to potentially add upto 32 memory regions,
and then also allow for refilling when serving alloc requests ($64-32=32$).
Why a reserve of 32 is needed is explained later in \ref{mem-con}.

The memory manager is now initialised to a point where we can pass it memory
regions via \verb|lib/mm/mm.c:mm_add|.
Compared to free, there two additionally thing to be considered when adding the
memory regions to the memory manager compared to a normal free of a memory
range.
They are not yet part of the \verb|all| double linked list.
This list contains all memory range in ascending order of their base address.
When adding memory regions them to it, we need to consider that the memory
ranges are not necessarily added in ascending order of their base address.
Additionally, from memory regions are the origin nodes (\verb|bi_node|) created
and added it to the \verb|bi| double linked list.
After having done that, the ram cap from the boot info has been transformed into
an ordinary memory range, which references the origin node, and can be added to
the free memory ranges using \verb|lib/mm/mm.c:mm_add_to_free|.

With all this done, the memory manager can now serve alloc and free requests.

\section{Datastructures} \label{mem-data}

To understand how alloc and free requests are served we introduce the
datastructures used in the memory manager.

\subsection{History}

Originally, everything has been implemented using two double linked lists (can
be seen on \verb|milestone2|).
One for \verb|struct capnode|, which stored the origin ram capabilities.
And one for \verb|struct mmnode|, which stored the memory ranges.
The memory ranges are stored in ascending order, and all of them, irrespective
if free or allocated.
Every memory range has a reference to a capnode.

When now \verb|lib/mm/mm.c:mm_alloc_aligned| is called, we would search the
mmnode linked list from the beginning, until we found a free node which is
greater than the size requested.
We split the mmnode from the left.
As free mmnode don't have a capability yet created for them, we don't need to
call \verb|cap_destroy|, but can go right for \verb|cap_retype|.
For this we use the reference to the capnode, because the capnode has the
original base, and so the \verb|cap_retype| is quite easy to do, as there are
no indirections, all ram capabilities handed out are direct descendants.

For \verb|lib/mm/mm.c:mm_free|, we again search the mmnode linked list, and we
expect to find an allocated mmnode with the base address given to free.
We mark this node simply as free.

Both of these operations have a cost of $\mathcal{O}(n)$, so this way of
implementing it is quite expensive.
Additionally, any external fragmentation isn't undone, even though the
datastructures have been setup to make this easy, by having memory ranges in
ascending order and using doubly linked list.

\subsection{Current State}

The current state still hase a double linked list of all memory ranges in
ascending order. This to allow for easy splitting (insertion) and fusing of
memory ranges (deletion), which are both $\mathcal{O}(1)$ operations on a doubly
linked list.

For allocation, we change from using the first memory range that's larger than
the requested size, to using the memory range with least size, which is greater
than the requested size.

As in \verb|lib/collections| the is no datastructure with allows for this
request to be fulfilled (hash table only allows for lookup of known keys), we
implemented an avl tree.
We chose the avl tree because we know it from the lectures, and didn't find
convincing literature that for example a red black tree is significantly better.
% TODO: Potentially add Performance Analysis of BSTs in System Software as
% source
The implementation of the avl tree is done so it doesn't allocate memory itself
(important later on, \ref{mem-con}).
If using slab allocator per avl tree, the avl nodes shouldn't be too dispersed
in memory, in theory making the pointer chasing somewhat cache friendly.
Also, it's versatile, in that it just stores opaque pointers (\verb|void *|),
allowing for reuse later on.
Additionally, when we have a reference to the specific avl node, we save one
lookup, making things a bit faster but the implementation also more complicated,
because we need to swap nodes under the hood (easy to get wrong).

So for allocation we have an avl tree that is indexed by the size of a memory
range, and as nodes double linked lists of memory ranges of that size (as it
can be that multiple free memory ranges have the same size).
Potentially the memory range is still to large, so we split it and add back to
the free avl tree ($\mathcal{O}(log(n))$), and insert into the all doubly linked
list ($\mathcal{O}(1)$).
As any of these memory range is ok, removing from this list is $\mathcal{0}(1)$.
With this, allocation is now an $\mathcal{O}(log(n))$ operation, which for worst
case performance is the theoretical optimum (comparison based search).

For freeing we need another avl tree indexed by the base address, the allocated
avl tree.
There is only ever one memory range with a certain base address, so here we
don't need a list as node.
We remove the memory range from the allocated tree.
Having a doubly linked list of memory ranges in ascending orders, and references
to the capnodes comes in handy now.
We can just check if the left neighbour is also free, and has same origin, and
fuse them together in $\mathcal{O}(log(n))$ (we need to reinsert into avl free
tree).
Vice versa for the right neighbour.

To note in all of this is that we do only one
\verb|cap_retype|/\verb|cap_destroy| per alloc/free, minimizing context
switches.
E.g. free memory ranges don't have their ram capabilites eagerly, because they
might be fused later on.
And additionally there is no hierarchy of ram capabilities to keep track of,
making \verb|cap_retype| simple to implement, just subtract the base of capnode
and one is set.

% TODO: Potentially more material
% 
% Datastructures of memory manager:
% \begin{itemize}
% 	\item Linked list of boot info ram chunks, provides the capabilities off which
% 				everything else is split, not intermediaries
% 	\item Linked list of all memory ranges, in ascending order of adderss
% 	\item Avl tree of free memory ranges, indexed by their size
% 	\item Avl tree of allocated memory ranges, indexed by their address
% \end{itemize}
% 
% The linked list of all nodes in ascending order of address can be managed with
% O(1) operations, and allows for easy fusing of memory ranges, if for example an
% allocated memory range gets freed, and there is an adjacent free on.
% 
% The avl tree of allocatee memory ranges, indexed by their address, is need
% because we only get the adress to free, but not the size, so thats for
% remembering the size of a certain memory range.
% 
% The avl tree of free memory ranges, indexed by ther size, allows for fast
% lookup when allocating memory, one just for the least element that's greater or
% euqal to the size requested.
% 
% Steps taken in implementation:
% \begin{itemize}
% 	\item Used multiple linked list initially to manage memory.
% 				Which was actually not too bad, because if memory is not freed, we can
% 				just chop off of the large object, having allocation in O(1), better than
% 				with trees (But only if the threes are badly done, as in this case the
% 				free avl tree also only contains one node, is basically also O(1)).
% 				Freeing is whole other story, as searching for the address to free is O(n).
% 	\item Changed later on to avl tree, when paging became too slow, memory was first to switch
% 	\item Fuses memory with adjacent free memory upon return
% \end{itemize}

\section{Fragmentation}

Every allocation request is rounded up to BASE\_PAGE\_SIZE.
This has multiple reasons.
Most ram will get mapped, and for it be able to be mapped, it must be multiple
of BASE\_PAGE\_SIZE.
So that's why we don't expect many requests that are note a multiple of
BASE\_PAGE\_SIZE, and therefore not having too much internal fragmentation.
Also, we observed that we only get \verb|lib/mm/mm.c:alloc_aligned| requests for
aligments smaller or equal to \verb|BASE_PAGE_SIZE|.
By having allocations being a mulitple of \verb|BASSE_PAGE_SIZE|, we don't
need to concern us further with alignment restrictions, as these are
automatically fulfilled.

For external fragmentation, we try to undo it by fusing adjacent free memory
ranges.
We try to use the smallest memory range that is large enough to fulfill the
memory request, and splitting off from the left, leaving as much of the free
memory rang intact.
This makes answering if there is a large enough memory range simple, as what
can be fused is fused.
We did not further investigate if this is detrimental to external fragmentation.

% TODO: Potentially more material
% 
% Fragmentation:
% \begin{itemize}
% 	\item Allocations always multiples of BASE\_PAGE\_SIZE
% 	\begin{itemize}
% 		\item Easy to fulfill requests with alignment <= BASE\_PAGE\_SIZE
% 		\item Reduces number of differen size held in avl tree of free memory ranges
% 		\item Should also reduce the amount of external fragmentation into the system
% 	\end{itemize}
% \end{itemize}
% 
% % We painted us a bit in a corner by not requiring to have capability as an
% % argument to mm_free
% Problems with lib/aos/capabilites.c:frame\_create, because it destroys the ram\_cap
% immediately. Not really, we could just as well have passed through frame cap.
% 
% Memory server, deletes cap as soon as sent off, not requiring cap makes for lighter
% free.

\section{Reentrancy and Concurrency}\label{mem-con}

The memory manager needs to be able to pull himself out of a ditch by it's own
hair (see Münchhausen).
To do that, it needs to always have enough ressources in reserve to build out
its datastructures without needing to build out its datastructures.

That's we use separate \verb|slab_allocator|'s, which we refill when we reach a
certain watermark.
A watermark of 32 nodes each for the tree and all list has so far proven to be
enough.
We werent able to verify this reserve staticly because of the complexity of the
\verb|lib/aos/slot_alloc| construct, especially with it's
\verb|twolevel_slot_alloc.c|.
That's also the reason why we only use the global slot\_allocator, not like with
the slabs, where we have separate slabs for the memory manager.
via \verb|lib/mm/mm.c:mm_add|.
Also that watermark refilling needs to be locked, and needs to be less or equal,
or else the refilling might not trigger early enough, or multiple times.
For the \verb|slot_allocator|, we assume it itself keeps enough reserves to build
itself, as it doesn't provide any method to check it's free slots, allows for
explicit refill.

Also, to build out it's datastructures, it needs to call \verb|ram_alloc| while
it's still in a ram alloc.
E.g. the code needs to handle reentrancy, a form of concurrency, correctly.

For that we need to identify sections of code, so called critical sections, that
can't handle being preemted, another request running through concurrently, then
resuming.
This is for example the case when we for alloc have a suiting avl node, then
would get preemted, get that suiting avl node snatched away, but we being
oblivious to that, stil continue with that avl node, corrupting the
datastructures.

Having identified the critical sections, we move out all calls that might to
reentrancy, for an isr that would be equivalent to disable interrupts.
That's why having avl trees not manage their own memory is so critical,
otherwise we wouldn't be able to write correct code.
To make sure we didn't miss a call, and to explicitly state this invariant, the
critical sections are guarded by the \verb|alloc_mutex|.
We call this type of not explicitly needed mutexes a canary mutex, and like a
canary in a mine, it would warn us if something changed in the system that we
did not anticipate, by leading to a deadlock, locking up the system.
It would also have been nice if lib/aos/thread\_sync.c:thread\_mutex\_lock would
have an assert checking if the same thread id retakes the lock, which would
always be a deadlock condition.

% --- End ---
% 
% A reserve of 32 has so far deemed to be large enough to serve every
% slab/slot\_allocator refill recursion loop we have encountered
% 
% Slab\_alloc can't trigger reentrancy, because it is explicitly filled. Slot\_alloc
% in contrast can trigger unwanted reentrancy.
% 
% Have lock which guarantees invariant that critical sections are reentrancy free.
% If not, we get a deadlock, and have an early warning that our critical section
% is not reentrancy free anymore.

\section{Limitations}

We only support for at most one memory region coming from the boot info.

In case of an error, we might not fully clean up after us.
This shouldn't invalidate our invariants, but might reduce the amount of memory
effectively usable.

We do not get a ram cap when freeing, which was already so in the original hand
out.
So we are not sure when freeing ram if the one freeing is really owner of the
freed memory range.

The implementation should work correctly until \verb|slab_default_refill| errors,
because the code assumes \verb|slab_alloc| always works, this would trigger an
assert on a later alloc.
