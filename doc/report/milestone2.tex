\chapter{Processes, threads, and dispatch}

Before the Barrelfish kernel will spawn a process for us, a lot of preparation in userspace
has to be done. We have to create the CSpace and virtual address space for the new domain, load
the binary and setup the dispatcher. Whereas, all tools needed for creating the CSpace where provided
by the handout, we had to extend the paging system to create the VSpace for the child. 

At the beginning of this milestone the paging was able to map a frame at a given virtual address. 
Our first task was to extend the paging code such that we can map a frame at any free virtual address.
All important changes and design decision about paging are documented in section \ref{sec:paging} and
are not discussed here. The only relevant design decision for this chapter is, that the paging library
doesn't handle unaligned addresses. This was a deliberate choice to make the client actively think about
what they actually map. One of these clients (a certain group member) wasn't thinking hard enough about
his mappings, which caused a longer debugging session, but more on that later.

Although, the workload of this milestone was quite heavy, there weren't much design decisions involved.
All we had to do was following a receipt and spawn a process in the end. For that reason is this chapter
relatively short.

After we extended the paging code we had to locate the ELF binary. This was relatively easy as most code
was given by the handout. Now that we have a frame capability to the binary, it is time to map it into
\verb|init|'s address space using the recently implemented \verb|paging_map_frame_attr()|.

Next we had to setup the CSpace of the new domain. This was pretty straight forward, although the whole
capability system was a bit intimidating at first sight.

In order to setup the virtual address space for the child, we create a new paging state with 
\verb|paging_init_state_foreign|. We extended this function with the argument \verb|max_addr| to specify
an upper limit on the virtual address space. The child doesn't know anything about the VSpace, which
\verb|init| had created before (passing the paging state is marked as extra challenge). Thus, we use
only addresses from from slot 0 of the level 0 page table to setup the VSpace in \verb|init|. We can set
this limit with the \verb|max_addr| argument. When the child starts running, the paging is initialized to
hand out addresses from slot 1 of the level 0 page table onward. This prevents any conflicts, with the 
mappings setup by \verb|init|.

Now that we have paging state for the child, we can start mapping frames into the child's address space.
Most mappings were straightforward, only the mapping of the ELF segments caused some troubles. The 
paging library expects correctly aligned addresses from the client and the addresses for the ELF sections
where obviously not correctly aligned. This was one reason why we initially failed to spawn a process, but
after clearly communicating this requirement, we mapped the ELF segments at the correct location.

The next bit was parsing the commandline arguments. We did just the standard parsing without any sophisticated
features like escaping. Then we initialized the dispatcher structure, set the correct entry point and finally
called \verb|armv8_set_registers()| to spawn the process and, of course, nothing happened. First we fixed the
issue with the ELF segments and then debugged the kernel to find out why the new process wasn't coming up. It
turned out that we had to set register \verb|x0| to the address of the argument frame and after we did that
the process finally printed a \verb|Hello, world! from userspace|.

It is not mentioned anywhere in the book, that we have to set register \verb|x0| correctly, so we are not
sure if this was omitted deliberately or by mistake. Anyway, it was a valuable lesson to debug in a very
limited environment like kernel.

