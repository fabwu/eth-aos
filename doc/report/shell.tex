\chapter{Shell}

The shell is a command-line interface which provides access to commands and programs on a system.
It makes the system interactive, such that it takes input from the user and performs actions accordingly.

This project consisted of two parts.
The first was the infrastructure for running the shell, specifically running the UART driver in userspace, reading from the console, and enabling access to the console over RPCs.
This turned out to be the more challenging part.
The second part was the shell itself, namely parsing a line, running programs and executing built-in commands.

In the following sections we will walk over each of these steps, describing the design and challenges in each.


\section{Userspace UART driver}

In order to allow reading characters from the UART, the first step was to enable the userspace UART driver.
The driver itself was already provided in the handout.

Enabling the driver was fairly straightforward. It consisted of the following steps:
\begin{itemize}

    \item
        Map the device registers into the virtual address space of the process.
        The \verb|DevFrame| capability required for device access was already present in \verb|init| (in the \verb|TASKCN_SLOT_DEV| slot).
        The capability could also be requested by other processes using the \verb|aos_rpc_get_device_cap| RPC call (already implemented as part of the filesystem project).
        The physical address of the LPUART3 register was provided in a header file.
        We mapped the device memory as read-write non-cacheable, to ensure that all character reads and writes would access the UART device immediately.

    \item
        Call the UART driver's initialization function (\verb|lpuart_init|).

\end{itemize}


\section{Character input}

The next step was reading characters from the UART.

The function provided by the UART driver (\verb|lpuart_getchar|) was non-blocking, returning when a character was not available.
Polling the driver would however block the rest of the system.
Fortunately the UART device can deliver an interrupt when a character is typed.
This enables the characters to be processed in an event-driven manner, without polling.

To use the interrupt, we needed to additionally configure the ARM Generic Interrupt Controller (GIC).
The userspace driver for the GIC was provided in the handout.

Enabling interrupt-driven character processing consisted of the following steps:

\begin{itemize}

    \item
        Map the GIC device registers. This was similar to mapping the UART device registers (see above).

    \item
        Initialize the GIC by calling the driver's \verb|gic_dist_init| function.

    \item
        Obtain an interrupt capability for the UART's interrupt line.
        This could be done via the \verb|inthandler_alloc_dest_irq_cap| function.
        In order to obtain the capability, the domain needed to additionally have access to the \verb|IRQTable| capability.
        While the capability is passed to \verb|init| in the Task CNode, it is not passed to other domains.
        We modified the spawning code to also pass it to the terminal domain.

    \item
        Register an interrupt handler using \verb|inthandler_setup|.
        This creates an LMP endpoint, ensuring that a message is delivered to the terminal domain whenever a UART interrupt occurs.
        We selected a high interrupt priority level, as with the lowest levels the interrupt did not get delivered.

    \item
        Enable the interrupt in the GIC hardware (using \verb|gic_dist_enable_interrupt|).

    \item
        Enable the interrupt in the UART hardware (using \verb|lpuart_enable_interrupt|).

\end{itemize}

With the interrupt controller configured, we were able to read characters from the UART whenever a key was pressed.

We quickly noticed that pressing multiple keys quickly in sequence would cause no more characters to be delivered.
This was because the interrupt only fired when the UART character buffer changed between empty and non-empty state.
Therefore, if one character arrived while another was being read by the driver, the buffer would be left non-empty, and no more interrupts would be triggered.
The correct approach was to read all characters out of the buffer whenever the interrupt arrived.


\section{Accessing the serial driver from other processes}

The serial driver needs to be accessible from any process, including the shell.
Access is provided through the \verb|aos_rpc_serial_getchar| and \verb|aos_rpc_serial_putchar| RPC calls, which respectively read and write one character.

A central question is whether the serial driver should run in a separate domain or within the \verb|init| process.
While a separate domain would be better for modularity, it requires routing to and from the driver.
For simplicity, we chose to run the driver in \verb|init| (but see Section~\ref{shell-integration-ns} for efforts to integrate with the nameserver).

The driver ran as part of \verb|init| on core 0.
We enabled writing a character from any process on the same core (via LMP) or another core (via UMP).
Due to time constraints, reading a character was only enabled on core 0.


\subsection{Reading a character}

Reading a character was done on a line-by-line basis.
Upon interrupt arrival, characters are stored in a buffer.
When a newline is typed, the first character in the buffer is sent to the currently registered domain.

When a \verb|getchar| request arrives, the channel of the requesting domain is saved.
No RPC response is sent, which ensures that the domain is blocked on the call, and does not need to poll.
As mentioned, when a newline is typed, the first character in the buffer is sent to the domain (using the saved channel).
The domain then executes further \verb|aos_rpc_serial_getchar| requests to read the rest of the line.

There are multiple corner cases to consider.
When multiple domains execute an \verb|aos_rpc_serial_getchar| request, then the last one gets the characters.
It would be better to have some kind of line-based session management.
Secondly, while a line is being read, the serial driver does not allow any further characters to be typed, which may be a limitation if the line is long.
We (arbitrarily) chose to allow lines up to one page in size (4096 bytes).
If the input gets longer, the entire line is simply dropped.
We also had to support an empty line being typed (just pressing Enter).

\subsection{Writing a character}

Writing a character was also done on a line-by-line basis.
Characters are buffered until a newline is typed (or the buffer capacity is reached).
This allows to have per-process line buffers, which ensures that prints from different processes are not mixed.
We implemented the buffers with dynamic allocation and a linked list to keep track of the buffers.
This also required the \verb|aos_rpc_serial_putchar| call to take the sending domain ID as an argument.


\section{C library integration}

While \verb|putchar| and \verb|getchar| are necessary building blocks, it would not be reasonable to expect applications to read or write one character at a time.
For this reason, a wrapper for reading or writing a line (or any string) was required.
While we could have added our own wrapper, the C library already supports functions which also do string formatting and other useful things.
The C library also supports callbacks to be registered with it.
By registering our functions (\verb|aos_terminal_read| and \verb|aos_terminal_write|) with the C library, applications (including the shell) were able to use the standard \verb|printf| and \verb|fgets| functions to read and write using the userspace serial driver.


\section{The shell}

We implemented the shell as a standalone process.
After a line of input arrives, the shell parses it into tokens (using the handy \verb|make_argv| function provided in the handout) and executes commands based on it.
We implemented several built-in commands as well as the ability to run processes:

\begin{itemize}

    \item
        \textbf{echo}. Prints a line.

    \item
        \textbf{help}. Lists all possible commands and how to run them.

    \item 
        \textbf{led (on, off)}. Turns an LED on or off.

    \item
        \textbf{ps}. Lists all currently running processes (name, PID, core ID), fetching the information from the process manager (\verb|init| on core 0) using the \verb|aos_rpc_process_get_all_pids| and \verb|aos_rpc_process_get_name| RPC calls.

    \item
        \textbf{time}. Times a command using the \verb|systime_now| function. Unfortunately due to time constraints we were not able to implement the timing of processes (only built-in commands), which makes the command significantly less useful for performance measurements.

    \item
        \textbf{Create processes}. As running a program is very common, then whenever a typed command is not recognized as a built-in command, the shell tries to start a process with that name. A process is created by performing the \verb|aos_rpc_process_spawn| RPC call to \verb|init|, which starts the process on core 0. Due to time constraints we were only able to spawn processes from the multiboot image, not from the SD card.

    \item
        \textbf{oncore}. Creates a process as above, except allows to specify the core ID to start the process on.

    \item
        \textbf{ls, mkdir, rmdir, touch, cat}. Enables to access and modify the filesystem on the SD card.

    \item
        \textbf{udpecho, arp}. Starts a UDP echo server and prints ARP tables used in networking.

\end{itemize}


\section{Bonus tasks}

Unfortunately due to time constraints we were not able to implement the I/O redirection or network login.


\section{Additional challenges}

During the course of implementing this project, several challenges had to be overcome.
While we have already described most of them, a few are not directly related to the shell.
Here we describe the remaining challenges and how they were solved.

\subsection{Running multiple processes on the same core}

The first challenge appeared before the project had even started.
As it turned out, when we ran multiple processes on the same core, one of them would get stuck.
Until now we had always ran one process at a time, but now we needed to run both the nameserver and the shell.

After a lengthy debugging session, we discovered that something was going wrong when both processes performed RPCs at the same time.
Specifically, whenever we were not able to send an LMP message immediately (because the other process was sending lots of messages), we would call \verb|lmp_chan_register_send()| to register a closure for sending later.
However, we found that the send closure was never being called.

We eventually discovered that the closure would only be called when the process was scheduled again.
However, in our RPC waiting code we were calling \verb|event_dispatch()|.
As it turned out, \verb|event_dispatch()| puts a process to sleep until an event arrives (such as an LMP message), and until then does not schedule a process again.
Therefore the send closure was not called again either.

To fix the issue, we switched our RPC waiting code to use \verb|event_dispatch_non_block()| followed by \verb|thread_yield()|.
This ensured that the process would get scheduled again, the closure would be called, and the process would not get stuck.

Alternatively, a call to \verb|thread_yield()| would have been enough, as there was no event to wait for.
However our waiting code was shared by the send and receive handlers, and using \verb|event_dispatch_non_block()| ensured that it worked for both (albeit with a likely performance penalty).

\subsection{Integration with the nameserver}
\label{shell-integration-ns}

While we did successfully implement a nameserver, the serial driver was unfortunately not integrated with it.
Instead, the serial driver was run as part of the \verb|init| process.
There were several reasons for this.

First, the serial driver posed the unique challenge that it required one RPC call (a \verb|getchar| request from one domain) to block while other RPC calls (\verb|putchar| requests from other domains) were serviced.
As the nameserver processed all requests to a domain/service over one channel, this was not possible.
Unfortunately we discovered this too late.
Had we had more time, we would have made it possible to allow a domain to register services over multiple channels.

Second, the nameserver was implemented to pass each message over a large frame, to allow quick transfer of large amounts of data.
However, for \verb|putchar| and \verb|getchar| requests, only a single byte was being passed, leading to a large performance overhead from frame allocation.
Had we had more time, we would have optimized the nameserver to allocate less memory for small messages.


\section{Limitations}

There are several things that could have been done better had we had more time:

For the \textbf{serial driver}:
\begin{itemize}

    \item
        Support for control characters, such as backspace/delete, or moving within a line, were frequently requested by team members.

\end{itemize}

For the \textbf{RPC} interface:
\begin{itemize}

    \item
        It would have been good to support multiple processes reading a line, including switching input to another process when the shell starts it.

    \item
        A \verb|flush| RPC would have enabled a better command line prompt (as otherwise only a newline triggered printing).

\end{itemize}

For the \textbf{shell}:
\begin{itemize}

    \item
        Additional commands would have been useful, such as setting a process to run in the background.

\end{itemize}

