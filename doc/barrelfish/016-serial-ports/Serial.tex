\documentclass[a4paper,twoside]{report} % for a report (default)

\usepackage{amsmath}
\usepackage{bftn} % You need this
\usepackage{cite}
\usepackage{color}
\usepackage{listings}
\usepackage{xspace}

\title{Serial ports in Barrelfish}   % title of report
\author{Timothy Roscoe}	% author
\tnnumber{016}  % give the number of the tech report
\tnkey{Serial ports} % Short title, will appear in footer

% \date{Month Year} % Not needed - will be taken from version history

\begin{document}
\maketitle

%
% Include version history first
%
\begin{versionhistory}
\vhEntry{0.0}{08.07.2012}{TR}{Initial version}
\end{versionhistory}

% \intro{Abstract}		% Insert abstract here
% \intro{Acknowledgements}	% Uncomment (if needed) for acknowledgements
% \tableofcontents		% Uncomment (if needed) for final draft
% \listoffigures		% Uncomment (if needed) for final draft
% \listoftables			% Uncomment (if needed) for final draft

\lstset{
  language=C,
  basicstyle=\ttfamily \small,
  flexiblecolumns=false,
  basewidth={0.5em,0.45em},
  boxpos=t,
}

\newcommand{\note}[1]{[\textcolor{red}{\emph{#1}}]}
\newcommand{\Intf}{\texttt{/kernel/include/serial.h}\xspace}

\tableofcontents

%%%%%%%%%%%%%%%%%%%%%%%%%%%%%%%%%%%%%%%%%%%%%%%%%%%%%%%%%%%%%%%%%%%%%%%%
\chapter{Introduction}

This technical note describes the use of serial ports (UARTS) in
Barrelfish.  

At present, it only describes the kernel-space interface
to serial ports, used internally by CPU drivers for debugging and
console output.  

In the future, it will also describe the interface to user-space UART
drivers and other console subsystems. 

\chapter{CPU driver interface}\label{cha:cpudriver}

CPU drivers use serial port access for printing debug information,
status messages during startup, and connecting a debugger such as GDB
to a processor. 

Each CPU driver implements a serial port subsystem which provides a
small number serial ports to the rest of the CPU driver.  

The interface to this subsystem is specified in the file
\Intf, and is described in this chapter.
Check the header file for a (possibly) more-up-to-date description. 

The interface is intentionally very basic.  It assumes each UART runs
at a fixed, default speed with basic settings (typically 115000 baud,
8-bit words, no parity, no flow control).   Serious applications
should use a full user-space UART driver instead. 

\section{Physical ports}

Serial ports available inside a CPU driver are numbered starting at
zero. 

The number of available serial ports is given by:
\begin{alltt}
  extern const unsigned serial\_num\_physical\_ports;
\end{alltt}

\subsection{Initialization}

Each port must be initialized before it can be used:
\begin{alltt}
  extern errval\_t serial\_init(unsigned port);
\end{alltt}
This call should fail with \texttt{SYS\_ERR\_SERIAL\_PORT\_INVALID} if
the port cannot be initialized, or the port number is out of range.

Initializing a serial port more than once, however, is permissible:
subsequent attempts to initialize an initialized port are silently
ignored.  The motivation for this is to handle the case where the
console and debug logical ports share the same physical port. 

In some cases it is necessary to do an ``early'' initializaton, for
example while running in physical address space before the MMU is
enabled.  In such cases the CPU driver should call, and the code
implement, the following:
\begin{alltt}
  extern errval\_t serial\_early\_init(unsigned port);
\end{alltt}

This returns the same errors as \texttt{serial\_init()}, but it is a
bug to call this function more than once with the same port number,
and doing so will cause an assertion failure. 

\subsection{Input/output}

Each serial port provides polled, blocking, single-character
programmed I/O:
\begin{alltt}
  extern void serial\_putchar(unsigned port, char c);
  extern char serial\_getchar(unsigned port);
\end{alltt}
It is a bug to call these with a port number that is out of range
(i.e. one that is greater than or equal to
\texttt{serial\_num\_physical\_ports}), and doing so will cause an
assertion failure.  It may also cause an assertion failure to call
these functions on a serial port which has not yet been initialized
with a call to \texttt{serial\_init()} or
\texttt{serial\_early\_init()}. 

Note that implementations of these functions should not attempt to use
DMA or interrupts.  They should also not buffer any data in software
(though they may use a hardware FIFO in the UART if appropriate). 

\section{Logical ports}

Barrelfish CPU drivers have access to two logical serial ports: 
\begin{itemize}
\item The \emph{console} port is used for general logging and debug
  information.  It is principally called by \texttt{kputchar()}, which
  implements line-buffering and is in turn called by the kernel
  \texttt{printf()} function.
\item The \emph{debug} port can be used to attach an external debugger
  like \texttt{gdb}. 
\end{itemize}

Both logical ports are assigned to sensible default physical ports by
the serial port implementation, but they can be changed by writing
these variables:
\begin{alltt}
  extern unsigned serial\_console\_port;
  extern unsigned serial\_debug\_port;
\end{alltt}

Note that it is not unusual for both logical ports to be assigned to
the same physical port. 

The file \Intf also includes inline functions to initialize whichever
physical port is currently assigned to a logical port:
\begin{alltt}
  static inline errval\_t serial\_console\_init(void);
  static inline errval\_t serial\_debug\_init(void);
\end{alltt}

There are also input functions, which simply call
\texttt{serial\_getchar()} with the appropriate port number:
\begin{alltt}
  static inline char serial\_console\_getchar(void);
  static inline char serial\_debug\_getchar(void);
\end{alltt}

Output functions are similar, but in addition replace any LF character
with the sequence CR/LF:
\begin{alltt}
  static inline void serial\_console\_putchar(char c);
  static inline void serial\_debug\_putchar(char c);
\end{alltt}

\section{Implementing a serial port subsystem}

Typically, a rudimentary console driver is the first thing to be
implemented when writing a CPU driver for a new architecture.   

The serial port subsystem for a CPU driver must implement all the
functionality in \Intf that deals with physical serial ports, in other
words:
\begin{verbatim}
  const unsigned serial_num_physical_ports;
  errval_t serial_init(unsigned port);
  errval_t serial_early_init(unsigned port);
  void serial_putchar(unsigned port, char c);
  char serial_getchar(unsigned port);
  unsigned serial_console_port;  
  unsigned serial_debug_port;
\end{verbatim}

Functions for logical ports are provided by inline functions in the
header file. 

Initializing a serial port typically consists of configuring the UART
hardware, and then mapping the registers appropriately into the
kernel's virtual address space. 

Examples of serial port implementations at time of writing include:
\begin{itemize}
\item \texttt{/kernel/arch/omap44xx/omap\_uart.c}
\item \texttt{/kernel/arch/x86/serial.c}
\end{itemize}

\subsection{Multiprocessor considerations}

This document does not address accessing a single UART from more than
CPU driver.  If this happens at all, it is typically hidden by the
implementation (for example, some x86 CPU drivers use a global
spinlock, since performance is not an issue anyway).  Care should be
taken not to initialize the UARTs multiple times in such
circumstances. 



\end{document}
